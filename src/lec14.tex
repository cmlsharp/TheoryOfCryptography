\documentclass[11pt]{article}

\usepackage{fullpage}
\usepackage{newtxtext}
\usepackage[T1]{fontenc}
\usepackage{hyperref,microtype,pdfsync}
\usepackage{amsmath,amsfonts,amssymb,amsthm}
\usepackage{mathtools}
\usepackage{fancyhdr}

\usepackage[noend]{algorithmic}
\usepackage{algorithm}
\usepackage{verbatim}

%%% BLACKBOARD SYMBOLS

% hyperref package defines C and G, undefined to avoid conflicts
\let\C\relax
\let\G\relax
\newcommand{\C}{\ensuremath{\mathbb{C}}}
\newcommand{\D}{\ensuremath{\mathbb{D}}}
\newcommand{\F}{\ensuremath{\mathbb{F}}}
\newcommand{\G}{\ensuremath{\mathbb{G}}}
\newcommand{\J}{\ensuremath{\mathbb{J}}}
\newcommand{\N}{\ensuremath{\mathbb{N}}}
\newcommand{\Q}{\ensuremath{\mathbb{Q}}}
\newcommand{\R}{\ensuremath{\mathbb{R}}}
\newcommand{\T}{\ensuremath{\mathbb{T}}}
\newcommand{\Z}{\ensuremath{\mathbb{Z}}}
\newcommand{\QR}{\ensuremath{\mathbb{QR}}}

\newcommand{\Zt}{\ensuremath{\Z_t}}
\newcommand{\Zp}{\ensuremath{\Z_p}}
\newcommand{\Zq}{\ensuremath{\Z_q}}
\newcommand{\ZN}{\ensuremath{\Z_N}}
\newcommand{\Zps}{\ensuremath{\Z_p^*}}
\newcommand{\ZNs}{\ensuremath{\Z_N^*}}
\newcommand{\JN}{\ensuremath{\J_N}}
\newcommand{\QRN}{\ensuremath{\QR_{N}}}
\newcommand{\QRp}{\ensuremath{\QR_{p}}}

%%% THEOREM COMMANDS

\theoremstyle{plain}            % following are "theorem" style
\newtheorem{theorem}{Theorem}[section]
\newtheorem{lemma}[theorem]{Lemma}
\newtheorem{corollary}[theorem]{Corollary}
\newtheorem{proposition}[theorem]{Proposition}
\newtheorem{claim}[theorem]{Claim}
\newtheorem{fact}[theorem]{Fact}

\theoremstyle{definition}       % following are def style
\newtheorem{definition}[theorem]{Definition}
\newtheorem{conjecture}[theorem]{Conjecture}
\newtheorem{example}[theorem]{Example}
\newtheorem{protocol}[theorem]{Protocol}

\theoremstyle{remark}           % following are remark style
\newtheorem{remark}[theorem]{Remark}
\newtheorem{note}[theorem]{Note}
\newtheorem{exercise}[theorem]{Exercise}

% equation numbering style
\numberwithin{equation}{section}

%%% GENERAL COMPUTING

\newcommand{\bit}{\ensuremath{\set{0,1}}}
\newcommand{\pmone}{\ensuremath{\set{-1,1}}}

% asymptotics
\DeclareMathOperator{\poly}{poly}
\DeclareMathOperator{\polylog}{polylog}
\DeclareMathOperator{\negl}{negl}
\newcommand{\Otil}{\ensuremath{\tilde{O}}}

% probability/distribution stuff
\DeclareMathOperator*{\E}{E}
\DeclareMathOperator*{\Var}{Var}

% sets in calligraphic type
\newcommand{\calD}{\ensuremath{\mathcal{D}}}
\newcommand{\calF}{\ensuremath{\mathcal{F}}}
\newcommand{\calG}{\ensuremath{\mathcal{G}}}
\newcommand{\calH}{\ensuremath{\mathcal{H}}}
\newcommand{\calX}{\ensuremath{\mathcal{X}}}
\newcommand{\calY}{\ensuremath{\mathcal{Y}}}

% types of indistinguishability
\newcommand{\compind}{\ensuremath{\stackrel{c}{\approx}}}
\newcommand{\statind}{\ensuremath{\stackrel{s}{\approx}}}
\newcommand{\perfind}{\ensuremath{\equiv}}

% font for general-purpose algorithms
\newcommand{\algo}[1]{\ensuremath{\mathsf{#1}}}
% font for general-purpose computational problems
\newcommand{\problem}[1]{\ensuremath{\mathsf{#1}}}
% font for complexity classes
\newcommand{\class}[1]{\ensuremath{\mathsf{#1}}}

% complexity classes and languages
\renewcommand{\P}{\class{P}}
\newcommand{\BPP}{\class{BPP}}
\newcommand{\NP}{\class{NP}}
\newcommand{\coNP}{\class{coNP}}
\newcommand{\AM}{\class{AM}}
\newcommand{\coAM}{\class{coAM}}
\newcommand{\IP}{\class{IP}}

%%% "LEFT-RIGHT" PAIRS OF SYMBOLS

%% NOTE: this requires \usepackage{mathtools} in the document preamble

% inner product
\DeclarePairedDelimiter\inner{\langle}{\rangle}
% absolute value
\DeclarePairedDelimiter\abs{\lvert}{\rvert}
% a set
\DeclarePairedDelimiter\set{\{}{\}}
% parens
\DeclarePairedDelimiter\parens{(}{)}
% tuple, alias for parens
\DeclarePairedDelimiter\tuple{(}{)}
% square brackets
\DeclarePairedDelimiter\bracks{[}{]}
% rounding off
\DeclarePairedDelimiter\round{\lfloor}{\rceil}
% floor function
\DeclarePairedDelimiter\floor{\lfloor}{\rfloor}
% ceiling function
\DeclarePairedDelimiter\ceil{\lceil}{\rceil}
% length of some vector, element
\DeclarePairedDelimiter\length{\lVert}{\rVert}
% "lifting" of a residue class
\DeclarePairedDelimiter\lift{\llbracket}{\rrbracket}
\DeclarePairedDelimiter\len{\lvert}{\rvert}

%%% CRYPTO-RELATED NOTATION

% KEYS AND RELATED

\newcommand{\key}[1]{\ensuremath{#1}}

\newcommand{\pk}{\key{pk}}
\newcommand{\vk}{\key{vk}}
\newcommand{\sk}{\key{sk}}
\newcommand{\mpk}{\key{mpk}}
\newcommand{\msk}{\key{msk}}
\newcommand{\fk}{\key{fk}}
\newcommand{\id}{id}
\newcommand{\keyspace}{\ensuremath{\mathcal{K}}}
\newcommand{\msgspace}{\ensuremath{\mathcal{M}}}
\newcommand{\ctspace}{\ensuremath{\mathcal{C}}}
\newcommand{\tagspace}{\ensuremath{\mathcal{T}}}
\newcommand{\idspace}{\ensuremath{\mathcal{ID}}}

\newcommand{\concat}{\ensuremath{\|}}

% GAMES

% advantage
\newcommand{\advan}{\ensuremath{\mathbf{Adv}}}

% different attack models
\newcommand{\attack}[1]{\ensuremath{\text{#1}}}

\newcommand{\atk}{\attack{atk}} % dummy attack
\newcommand{\indcpa}{\attack{ind-cpa}}
\newcommand{\indcca}{\attack{ind-cca}}
\newcommand{\anocpa}{\attack{ano-cpa}} % anonymous
\newcommand{\anocca}{\attack{ano-cca}}
\newcommand{\euacma}{\attack{eu-acma}} % forgery: adaptive chosen-message
\newcommand{\euscma}{\attack{eu-scma}} % forgery: static chosen-message
\newcommand{\suacma}{\attack{su-acma}} % strongly unforgeable

% ADVERSARIES
\newcommand{\attacker}[1]{\ensuremath{\mathcal{#1}}}

\newcommand{\Adv}{\attacker{A}}
\newcommand{\AdvA}{\attacker{A}}
\newcommand{\AdvB}{\attacker{B}}
\newcommand{\Dist}{\attacker{D}}
\newcommand{\Sim}{\attacker{S}}
\newcommand{\Ora}{\attacker{O}}
\newcommand{\Inv}{\attacker{I}}
\newcommand{\For}{\attacker{F}}

% CRYPTO SCHEMES

\newcommand{\scheme}[1]{\ensuremath{\text{#1}}}

% pseudorandom stuff
\newcommand{\prg}{\algo{PRG}}
\newcommand{\prf}{\algo{PRF}}
\newcommand{\prp}{\algo{PRP}}

% symmetric-key cryptosystem
\newcommand{\skc}{\scheme{SKC}}
\newcommand{\skcgen}{\algo{Gen}}
\newcommand{\skcenc}{\algo{Enc}}
\newcommand{\skcdec}{\algo{Dec}}

% public-key cryptosystem
\newcommand{\pkc}{\scheme{PKC}}
\newcommand{\pkcgen}{\algo{Gen}}
\newcommand{\pkcenc}{\algo{Enc}} % can also use \kemenc and \kemdec
\newcommand{\pkcdec}{\algo{Dec}}

% digital signatures
\newcommand{\sig}{\scheme{SIG}}
\newcommand{\siggen}{\algo{Gen}}
\newcommand{\sigsign}{\algo{Sign}}
\newcommand{\sigver}{\algo{Ver}}

% message authentication code
\newcommand{\mac}{\scheme{MAC}}
\newcommand{\macgen}{\algo{Gen}}
\newcommand{\mactag}{\algo{Tag}}
\newcommand{\macver}{\algo{Ver}}

% key-encapsulation mechanism
\newcommand{\kem}{\scheme{KEM}}
\newcommand{\kemgen}{\algo{Gen}}
\newcommand{\kemenc}{\algo{Encaps}}
\newcommand{\kemdec}{\algo{Decaps}}

% identity-based encryption
\newcommand{\ibe}{\scheme{IBE}}
\newcommand{\ibesetup}{\algo{Setup}}
\newcommand{\ibeext}{\algo{Ext}}
\newcommand{\ibeenc}{\algo{Enc}}
\newcommand{\ibedec}{\algo{Dec}}

% hierarchical IBE (as key encapsulation)
\newcommand{\hibe}{\scheme{HIBE}}
\newcommand{\hibesetup}{\algo{Setup}}
\newcommand{\hibeext}{\algo{Extract}}
\newcommand{\hibeenc}{\algo{Encaps}}
\newcommand{\hibedec}{\algo{Decaps}}

% binary tree encryption (as key encapsulation)
\newcommand{\bte}{\scheme{BTE}}
\newcommand{\btesetup}{\algo{Setup}}
\newcommand{\bteext}{\algo{Extract}}
\newcommand{\bteenc}{\algo{Encaps}}
\newcommand{\btedec}{\algo{Decaps}}

% trapdoor functions
\newcommand{\tdf}{\scheme{TDF}}
\newcommand{\tdfgen}{\algo{Gen}}
\newcommand{\tdfeval}{\algo{Eval}}
\newcommand{\tdfinv}{\algo{Invert}}
\newcommand{\tdfver}{\algo{Ver}}

%%% PROTOCOLS

\newcommand{\out}{\text{out}}
\newcommand{\view}{\text{view}}

%%% COMMANDS FOR LECTURES/HOMEWORKS

\newcommand{\lecheader}{%
  \chead{\large \textbf{Lecture \lecturenum\\\lecturetopic}}

  \lhead{\small \textbf{Theory of Cryptography}\\}

  \rhead{\small \textbf{Instructor:
      \href{http://www.eecs.umich.edu/~cpeikert/}{Chris Peikert}\\Scribe:
      \scribename}}

  \setlength{\headheight}{20pt}
  \setlength{\headsep}{16pt}
}


% VARIABLES

\newcommand{\lecturenum}{14}
\newcommand{\lecturetopic}{Random Oracle Signatures}
\newcommand{\scribename}{Steven Ehrlich}

% END OF VARIABLES

\lecheader

\pagestyle{plain}               % default: no special header

\begin{document}

\thispagestyle{fancy}           % first page should have special header

% LECTURE MATERIAL STARTS HERE

\section{Recap and Looking Ahead}
\label{sec:recap-looking-ahead}

Previously we demonstrated a signature scheme that was derived from
any one-way function, plus a collision-resistant hash family; this
latter assumption can be replaced with a slightly weaker primitive
that also follows from one-way functions.  This means that even though
they are a ``public-key'' object, signature schemes exist in
minicrypt.

Unfortunately our scheme was quite inefficient.  Now we'll consider an
alternative signature scheme that is efficient and simpler to analyse.
To do this, we'll rely on stronger assumptions: we will rest our
scheme on trapdoor permutations instead of just OWFs, and the security
analysis will be in an \emph{idealized} setting called the Random
Oracle Model.

\section{Random Oracle Model}
\label{sec:random-oracle-model}

A random oracle is a ``public truly random function:''
\begin{itemize}
\item Anyone can query it as a ``black box.''
\item It gives a \emph{consistent}, \emph{uniform}, and
  \emph{independent} random output for each distinct input.
\end{itemize}

Is a random oracle substantially different from a PRF?  On the
surface, no: a PRF can be queried, and gives consistent outputs that
appear uniformly random and independent.  However, the point of having
a random oracle is for it to be \emph{public}.  If the seed of a PRF
is revealed, then it loses all its security guarantees.

\subsection{Methodology}
\label{sec:ro-methodology}

\href{http://cseweb.ucsd.edu/users/mihir/papers/ro.html}{Bellare and
  Rogaway} introduced a methodology for using random oracles in
cryptography:
\begin{enumerate}
\item Design a scheme in the RO model: all algorithms (of the scheme,
  and the adversary) have ``black-box'' access to the oracle.
\item Analyze/prove security (in the RO model) by giving a security
  reduction.
\item \emph{Instantiate} the oracle with a ``sufficiently crazy''
  public hash function (such as SHA-1), and hope that this does not
  introduce any security vulnerabilities in the scheme.
\end{enumerate}

\subsection{Advantages and Caveats}
\label{sec:advantages-caveats}

Why does having a random oracle help?  In the RO model, security is
often easier to prove, even for very simple and efficient schemes.
Generally speaking, this is because the simulator/security reduction
has some extra flexibility: it gets to provide the view of the oracle
to the adversary.  This not only allows the simulator to know what
queries the adversary makes, but also to `control' (or `program') the
oracle outputs that the adversary sees.  We shall see both of these
properties used in the security proof below.

However, there is (sometimes heated) dispute over the value of a
security proof in the random oracle model.  The difficulty lies with
the \emph{instantiation} step.  Because its code is fixed and public,
a given hash function is \emph{not} a black box, nor does it give
uniformly random and independent outputs.  In fact,
\href{http://arxiv.org/abs/cs.CR/0010019}{Canetti, Goldreich, and
  Halevi} showed that there exists (under standard assumptions) a
signature scheme that is secure in the random oracle model, but which
becomes \emph{trivially breakable} with \emph{any} public
instantiation of the oracle!  In this sense, the random oracle
methodology does \emph{not} yield secure schemes in general.  However,
the problematic CGH signature scheme is very contrived, and would
never be proposed for practice.  Most signature schemes used in the
real world are much more ``natural,'' and don't seem to have obvious
(or even non-obvious) flaws when instantiated with strong hash
functions.

\section{A Signature Scheme in the RO Model}
\label{sec:signature-scheme-ro}

\href{https://wiki.cc.gatech.edu/theory/images/7/75/Lec10.pdf}{Recall
  the definition of a trapdoor permutation family:} it is a set
$\set{f_{s} \colon D_{s} \to D_{s}}$ of permutations where $f_{s}$ is
easy to invert with knowledge of the trapdoor (which is generated
along with the index $s$), and otherwise hard to invert on uniformly
random values $y \in D_{s}$.  The RSA and Rabin function families are
plausible trapdoor permutation candidates.

\href{https://wiki.cc.gatech.edu/theory/images/7/75/Lec12.pdf}{Previously},
we attempted to design a secure signature scheme based on trapdoor
permutations.  Our old scheme was simple: the public verification key
was a function $f_{s}$, the secret signing key was its trapdoor (also
produced by the function sampler $\algo{S}$), and the signature for a
message $m \in D_{s}$ was $f_{s}^{-1}(m)$.  Unfortunately, this
succumbs to a trivial attack: first pick some signature $\sigma \in
D_{s}$, then set $m=f_s(\sigma)$.  Then $(m,\sigma)$ is clearly is a
valid message/signature pair.

Fortunately, we can resurrect this scheme in the random oracle model,
with a small change: we first \emph{hash} the message using the
oracle, then invert the hash.  The following scheme $\sig$, often
called \emph{full-domain hash}, uses a trapdoor permutation family
$\set{f_{s}}$ (where the domain of every $f_{s}$ is $\bit^{n}$, for
convenience) and a function $H : \bit^{*} \to \bit^{n}$, modelled as a
random oracle.
\begin{itemize}
\item $\siggen^{H}$: choose $(s,t) \gets \algo{S}$ and output $\vk=s$,
  $\sk=t$.
\item $\sigsign^{H}_{t}(m \in \bit^{*})$: output $\sigma =
  f_s^{-1}\parens*{H(m)}$.
\item $\sigver^{H}_{s}(m,\sigma \in \bit^{n})$: accept if
  $f_s(\sigma)=H(m)$, otherwise reject.
\end{itemize}

Despite the similarly with our previous signature scheme, we cannot
apply the same kind of attack: after choosing $\sigma \in \bit^{n}$,
we would then have to find a message $m$ such that $H(m) =
f_{s}(\sigma)$.  Because $H$ is a random oracle, this could only be
done by brute-force search, which would take about $2^{n}$ attempts.

Of course, just because \emph{this} attack is thwarted does not mean
that our scheme is secure.  Happily, we can prove security in the
idealized model.

\begin{theorem}
  \label{thm:fdh-uf-cma}
  The $\sig$ scheme described above is strongly unforgeable under
  chosen-message attack (in the RO model), assuming that $\set{f_{s}}$
  is a TDP family.
\end{theorem}

\begin{proof}
  First note that for any oracle $H$, every message $m$ has exactly
  one signature, so to proving strong unforgeability it suffices to
  prove standard unforgeability.

  We want to design a simulator algorithm $\Sim(s,y)$ that uses a
  hypothetical forger $\For$ to find $x = f_{s}^{-1}(y)$, given $s$
  generated by $\algo{S}$ and uniformly random $y \gets \bit^{n}$.
  (Note that $y = f_{s}(x) \in \bit^{n}$ is uniformly random because
  $x \gets \bit^{n}$ is, and because $f_{s}$ is a permutation.)  The
  basic idea is that $\Sim$ will answer the random oracle queries
  (``program the oracle'') so that it knows the preimages (under
  $f_{s}$) of all but one of them, which it answers as its challenge
  value $y$.  This allows the simulator to answer signing queries for
  all but one message, which it hopes is the message that $\For$ will
  forge, in which case the forged signature is exactly
  $f_{s}^{-1}(y)$, as desired.

  We now proceed in more detail.  Without loss of generality, we may
  make some simplifying assumptions about $\For$.  In particular, we
  assume that:
  \begin{itemize}
  \item \emph{$\For$ makes at most $q=\poly(n)$ queries to $H$.}
    Because $\For$ runs in polynomial time, it can't make more than
    $\poly(n)$ queries no matter what its execution path is.
  \item \emph{$\For$ always outputs some (possibly invalid) forgery
      $(m^{*}, \sigma^{*})$.}  We can always ``wrap'' $\For$ in a
    program that outputs an arbitrary ``junk'' forgery if $\For$
    fails to.
  \item \emph{$\For$ always queries $H(m)$ (or $H(m^{*})$) before
      querying $\sigsign(m)$ (or outputting its forgery
      $(m^*,\sigma^*)$, respectively).}  We can wrap $\For$ in another
    algorithm that inserts the needed queries, and ignores the
    results.
  \item \emph{$\For$ never repeats any query to $H$.}  We can always
    wrap $\For$ in another algorithm that watches all of $\For$'s
    queries, and store all queries/response pairs in a table.  Before
    forwarding each query to the random oracle, it first checks its
    table and returns the answer if it is already known.  (We note
    that we can enforce this condition on either the forger's end, or
    in the simulator's responses, with the same effect.)
  \end{itemize}

  The forger $\For$ expects to perform a chosen-message attack against
  the scheme, which means that $\Sim$ must provide it with access to:
  \begin{enumerate}
  \item a verification key $\vk$;
  \item a signing oracle $\sigsign_{\sk}(\cdot)$;
  \item a random oracle $H$.
  \end{enumerate}

  Our simulator $\Sim(s,y)$ works as follows: it first chooses $i^*
  \gets [q]$ uniformly at random; this is a ``guess'' as to which
  query to $H$ will correspond to the message in the eventual forgery.
  It then invokes $\For$ on $\vk = s$, and simulates the oracles $H$
  and $\sigsign(\cdot)$ as follows:

  \begin{algorithm}
    \caption{Simulation of $H(\cdot)$.}
    \begin{algorithmic}[1]
      \STATE On $\For$'s $i$th query $m_{i} \in \bit^{*}$,
      \IF {$i=i^*$}
      \STATE Store $(\bot, y, m_{i^{*}})$ in an internal table.
      \RETURN $y_{i^*}=y$ to $\For$
      \ELSE 
      \STATE Choose $x_i\gets \bit^n$
      \STATE Let $y_i=f_s(x_i)$
      \STATE Store $(x_i,y_i,m_i)$ in the internal table.
      \RETURN $y_i$ to $\For$
      \ENDIF
    \end{algorithmic}
  \end{algorithm}

  \begin{algorithm}
  \caption{Simulation of $\sigsign(\cdot)$.}
  \begin{algorithmic}[1]
  \STATE On a query $m \in \bit^{*}$, look up $(x,y,m)$ in the
  internal table.
  \COMMENT{Such an entry exists because $\For$ always queries $H(m)$ before
    querying $\sigsign(m)$.}
  \IF {$x\neq \bot$} 
  \RETURN $x$ to $\For$
  \ELSE
  \STATE Abort the simulation and fail.
  \ENDIF
  \end{algorithmic}
  \end{algorithm}

  Finally, when $\For$ outputs its purported forgery $(m^*,\sigma^*)$,
  check that $m^*=m_{i^*}$.  If so, $\Sim$ outputs $\sigma^*$;
  otherwise, it fails.

  We shall analyze the simulation in the claims below.  By
  Lemma~\ref{lem:adv-Sim} below, we conclude that the existence of a
  forger of non-negligible advantage implies that we have a
  non-negligible advantage in inverting the trapdoor permutation, as
  desired.
\end{proof}

\subsection{Analysis}

\begin{lemma}
  \label{lem:adv-Sim}
  For the above simulation strategy,
  \[ \advan_{\scheme{TDP}}(\Sim) = \frac{1}{q} \cdot \advan_{\sig}(\For). \]
\end{lemma}

\noindent This fact follows from two claims:

\begin{claim}
  Conditioned on not failing, $\Sim(s,y)$ simulates the chosen-message
  attack perfectly, and hence outputs $f_{s}^{-1}(y)$ with probability
  $\advan_{\sig}(\For)$.
\end{claim}

\begin{proof}
  We first note that $\vk=s$ is distributed precisely as in $\sig$,
  because $\Sim$ is playing the TDP inversion game.  Next, $\Sim$
  simulates the oracle $H$ perfectly: every answer $y_{i}$ is
  uniformly random and independent, because it equals $f_{s}(x_{i})$
  for some uniformly random $x_{i} \in \bit^{n}$, and $f_{s}$ is a
  bijection.  (Also $y_{i^{*}} = y$ is uniformly random, because
  $\Sim$ is playing the TDP inversion game.)  Lastly, because $\Sim$
  did not fail, it answered every signing query with its unique valid
  signature.

  Now $\Sim$ does not fail exactly when $m^{*}$ is $\For$'s $i^{*}$th
  query to $H$, where $(m^{*}, \sigma^{*})$ is $\For$'s output
  forgery.  Observe that in this case, $\sigma^*$ is a valid signature
  if and only if $\sigma^*=f_{s^{-1}}(y)$.  Because $\Sim$ simulates
  the chosen-message attack perfectly, $\For$ outputs such
  $\sigma^{*}$ with probability $\advan_{\sig}(\For)$, as desired.
\end{proof}

\begin{claim}
  The probability of $\Sim$ not failing equals $1/q$.
\end{claim}

\begin{proof}
  By assumption, $\For$ must query $H$ on $m^{*}$ before outputting
  its forgery $(m^{*}, \sigma^{*})$.  By construction of $\Sim$, it
  does not fail exactly when $m^{*}$ is the $i^*$th query to $H$.
  Moreover, we claim that $m^{*}$ is the $i^{*}$th query to $H$ with
  probability exactly $1/q$.  This is because we can imagine an
  alternative experiment in which \emph{every} query to $\sigsign$ is
  answered correctly, and $i^{*}$ is chosen uniformly; in such an
  experiment, $i^{*}$ is independent of $\For$'s view an hence $m^{*}$
  is the $i^{*}$th query with probability $1/q$.
\end{proof}
    
\section{Variants}
\label{sec:variants}

Notice the factor of $\frac{1}{q}$ in the advantage of the simulator.
The reduction (from breaking the TDP to breaking the signature scheme)
is ``loose'' in the sense that the simulator's probability of
inverting is much less than the forger's probability of forging.  For
concrete security, this might mean that we need to use a large
security parameter for our TDP in order to achieve a small enough
forging probability for our signature scheme.  Also notice that $q$ is
the number of \emph{hash} queries, not signing queries.  In practice,
the number of times an attacker can compute the hash function is
probably much larger than the number of signatures to which it has
access.

People have tried to improve the ``quality'' of the reduction in a
number of ways.  One notable way, due to
\href{http://www.cs.umd.edu/~jkatz/papers/CCCS03_sigs.pdf}{Katz and
  Wang}, is to use ``claw-free'' pairs of TDPs; these are analogous to
collision-resistant hash functions.  In a claw-free family, the
function sampler generates two functions $f_{s_{0}}$ and $f_{s_{1}}$,
and the security property is that it should be hard to find a pair of
(not necessarily distinct) inputs $x_{0}, x_{1} \in \bit^{n}$ such
that $f_{s_{0}}(x_{0}) = f_{s_{1}}(x_{1})$.  As usual, there are
trapdoors that make it easy to invert the functions.  (The connection
to collision resistance is the following: define $f(b,x) =
f_{s_{b}}(x)$.  Then $f$ maps $n+1$ bits to $n$ bits, and a collision
in $f$ immediately gives us a claw in $f_{s_{0}}, f_{s_{1}}$; the
values of $b$ must be different because the functions $f_{s_{b}}$ are
permutations.)

We can tweak the signature scheme so that the public verification key
consists of two functions $f_{s_{0}}, f_{s_{1}}$, and the signing
algorithm on message $m$ chooses one of the functions at random and
inverts it on $H(m)$.\footnote{If the forger queries the signing
  oracle on the same message more than once, the signer should invert
  the same function every time; this can be enforced by keeping state
  or using a PRF.  Without this condition, it is easy to see that the
  signer will eventually give away a claw!}  In the security proof,
the simulator is attempting to find a claw in $f_{s_{0}}, f_{s_{1}}$.
When simulating the random oracle on each new message $m_{i}$, it
\emph{always} chooses a fresh $x_{i}$ and applies one of the two
functions (chosen at random) to compute and return $y_{i}$.  If the
forger later queries the signing oracle on $m_{i}$, the simulator can
always reply with a properly distributed signature.  (But it only
knows \emph{one} valid signature, which is why the real scheme must
also give away only one distinct signature for each message.)
Finally, when the forger outputs a forgery, half of the time it will
end up forging using the same function that the simulator used (so we
fail), and the other half of the time it finds a preimage under the
other function, yielding a claw.  So the simulator's advantage is
\emph{half} that of the forger, which is a drastic improvement!

\end{document}

%%% Local Variables: 
%%% mode: latex
%%% TeX-master: t
%%% End: 
