\documentclass[11pt]{article}

\usepackage{fullpage}
\usepackage{mathptmx}
\usepackage[T1]{fontenc}
\usepackage{hyperref,microtype,pdfsync}
\usepackage{amsmath,amsfonts,amssymb,amsthm}
\usepackage{mathtools}
\usepackage{fancyhdr}

%%% BLACKBOARD SYMBOLS

% hyperref package defines C and G, undefined to avoid conflicts
\let\C\relax
\let\G\relax
\newcommand{\C}{\ensuremath{\mathbb{C}}}
\newcommand{\D}{\ensuremath{\mathbb{D}}}
\newcommand{\F}{\ensuremath{\mathbb{F}}}
\newcommand{\G}{\ensuremath{\mathbb{G}}}
\newcommand{\J}{\ensuremath{\mathbb{J}}}
\newcommand{\N}{\ensuremath{\mathbb{N}}}
\newcommand{\Q}{\ensuremath{\mathbb{Q}}}
\newcommand{\R}{\ensuremath{\mathbb{R}}}
\newcommand{\T}{\ensuremath{\mathbb{T}}}
\newcommand{\Z}{\ensuremath{\mathbb{Z}}}
\newcommand{\QR}{\ensuremath{\mathbb{QR}}}

\newcommand{\Zt}{\ensuremath{\Z_t}}
\newcommand{\Zp}{\ensuremath{\Z_p}}
\newcommand{\Zq}{\ensuremath{\Z_q}}
\newcommand{\ZN}{\ensuremath{\Z_N}}
\newcommand{\Zps}{\ensuremath{\Z_p^*}}
\newcommand{\ZNs}{\ensuremath{\Z_N^*}}
\newcommand{\JN}{\ensuremath{\J_N}}
\newcommand{\QRN}{\ensuremath{\QR_{N}}}
\newcommand{\QRp}{\ensuremath{\QR_{p}}}

%%% THEOREM COMMANDS

\theoremstyle{plain}            % following are "theorem" style
\newtheorem{theorem}{Theorem}[section]
\newtheorem{lemma}[theorem]{Lemma}
\newtheorem{corollary}[theorem]{Corollary}
\newtheorem{proposition}[theorem]{Proposition}
\newtheorem{claim}[theorem]{Claim}
\newtheorem{fact}[theorem]{Fact}

\theoremstyle{definition}       % following are def style
\newtheorem{definition}[theorem]{Definition}
\newtheorem{conjecture}[theorem]{Conjecture}
\newtheorem{example}[theorem]{Example}
\newtheorem{protocol}[theorem]{Protocol}

\theoremstyle{remark}           % following are remark style
\newtheorem{remark}[theorem]{Remark}
\newtheorem{note}[theorem]{Note}
\newtheorem{exercise}[theorem]{Exercise}

% equation numbering style
\numberwithin{equation}{section}

%%% GENERAL COMPUTING

\newcommand{\bit}{\ensuremath{\set{0,1}}}
\newcommand{\pmone}{\ensuremath{\set{-1,1}}}

% asymptotics
\DeclareMathOperator{\poly}{poly}
\DeclareMathOperator{\polylog}{polylog}
\DeclareMathOperator{\negl}{negl}
\newcommand{\Otil}{\ensuremath{\tilde{O}}}

% probability/distribution stuff
\DeclareMathOperator*{\E}{E}
\DeclareMathOperator*{\Var}{Var}

% sets in calligraphic type
\newcommand{\calD}{\ensuremath{\mathcal{D}}}
\newcommand{\calF}{\ensuremath{\mathcal{F}}}
\newcommand{\calG}{\ensuremath{\mathcal{G}}}
\newcommand{\calH}{\ensuremath{\mathcal{H}}}
\newcommand{\calX}{\ensuremath{\mathcal{X}}}
\newcommand{\calY}{\ensuremath{\mathcal{Y}}}

% types of indistinguishability
\newcommand{\compind}{\ensuremath{\stackrel{c}{\approx}}}
\newcommand{\statind}{\ensuremath{\stackrel{s}{\approx}}}
\newcommand{\perfind}{\ensuremath{\equiv}}

% font for general-purpose algorithms
\newcommand{\algo}[1]{\ensuremath{\mathsf{#1}}}
% font for general-purpose computational problems
\newcommand{\problem}[1]{\ensuremath{\mathsf{#1}}}
% font for complexity classes
\newcommand{\class}[1]{\ensuremath{\mathsf{#1}}}

% complexity classes and languages
\renewcommand{\P}{\class{P}}
\newcommand{\BPP}{\class{BPP}}
\newcommand{\NP}{\class{NP}}
\newcommand{\coNP}{\class{coNP}}
\newcommand{\AM}{\class{AM}}
\newcommand{\coAM}{\class{coAM}}
\newcommand{\IP}{\class{IP}}

%%% "LEFT-RIGHT" PAIRS OF SYMBOLS

%% NOTE: this requires \usepackage{mathtools} in the document preamble

% inner product
\DeclarePairedDelimiter\inner{\langle}{\rangle}
% absolute value
\DeclarePairedDelimiter\abs{\lvert}{\rvert}
% a set
\DeclarePairedDelimiter\set{\{}{\}}
% parens
\DeclarePairedDelimiter\parens{(}{)}
% tuple, alias for parens
\DeclarePairedDelimiter\tuple{(}{)}
% square brackets
\DeclarePairedDelimiter\bracks{[}{]}
% rounding off
\DeclarePairedDelimiter\round{\lfloor}{\rceil}
% floor function
\DeclarePairedDelimiter\floor{\lfloor}{\rfloor}
% ceiling function
\DeclarePairedDelimiter\ceil{\lceil}{\rceil}
% length of some vector, element
\DeclarePairedDelimiter\length{\lVert}{\rVert}
% "lifting" of a residue class
\DeclarePairedDelimiter\lift{\llbracket}{\rrbracket}
\DeclarePairedDelimiter\len{\lvert}{\rvert}

%%% CRYPTO-RELATED NOTATION

% KEYS AND RELATED

\newcommand{\key}[1]{\ensuremath{#1}}

\newcommand{\pk}{\key{pk}}
\newcommand{\vk}{\key{vk}}
\newcommand{\sk}{\key{sk}}
\newcommand{\mpk}{\key{mpk}}
\newcommand{\msk}{\key{msk}}
\newcommand{\fk}{\key{fk}}
\newcommand{\id}{id}
\newcommand{\keyspace}{\ensuremath{\mathcal{K}}}
\newcommand{\msgspace}{\ensuremath{\mathcal{M}}}
\newcommand{\ctspace}{\ensuremath{\mathcal{C}}}
\newcommand{\tagspace}{\ensuremath{\mathcal{T}}}
\newcommand{\idspace}{\ensuremath{\mathcal{ID}}}

\newcommand{\concat}{\ensuremath{\|}}

% GAMES

% advantage
\newcommand{\advan}{\ensuremath{\mathbf{Adv}}}

% different attack models
\newcommand{\attack}[1]{\ensuremath{\text{#1}}}

\newcommand{\atk}{\attack{atk}} % dummy attack
\newcommand{\indcpa}{\attack{ind-cpa}}
\newcommand{\indcca}{\attack{ind-cca}}
\newcommand{\anocpa}{\attack{ano-cpa}} % anonymous
\newcommand{\anocca}{\attack{ano-cca}}
\newcommand{\euacma}{\attack{eu-acma}} % forgery: adaptive chosen-message
\newcommand{\euscma}{\attack{eu-scma}} % forgery: static chosen-message
\newcommand{\suacma}{\attack{su-acma}} % strongly unforgeable

% ADVERSARIES
\newcommand{\attacker}[1]{\ensuremath{\mathcal{#1}}}

\newcommand{\Adv}{\attacker{A}}
\newcommand{\AdvA}{\attacker{A}}
\newcommand{\AdvB}{\attacker{B}}
\newcommand{\Dist}{\attacker{D}}
\newcommand{\Sim}{\attacker{S}}
\newcommand{\Ora}{\attacker{O}}
\newcommand{\Inv}{\attacker{I}}
\newcommand{\For}{\attacker{F}}

% CRYPTO SCHEMES

\newcommand{\scheme}[1]{\ensuremath{\text{#1}}}

% pseudorandom stuff
\newcommand{\prg}{\algo{PRG}}
\newcommand{\prf}{\algo{PRF}}
\newcommand{\prp}{\algo{PRP}}

% symmetric-key cryptosystem
\newcommand{\skc}{\scheme{SKC}}
\newcommand{\skcgen}{\algo{Gen}}
\newcommand{\skcenc}{\algo{Enc}}
\newcommand{\skcdec}{\algo{Dec}}

% public-key cryptosystem
\newcommand{\pkc}{\scheme{PKC}}
\newcommand{\pkcgen}{\algo{Gen}}
\newcommand{\pkcenc}{\algo{Enc}} % can also use \kemenc and \kemdec
\newcommand{\pkcdec}{\algo{Dec}}

% digital signatures
\newcommand{\sig}{\scheme{SIG}}
\newcommand{\siggen}{\algo{Gen}}
\newcommand{\sigsign}{\algo{Sign}}
\newcommand{\sigver}{\algo{Ver}}

% message authentication code
\newcommand{\mac}{\scheme{MAC}}
\newcommand{\macgen}{\algo{Gen}}
\newcommand{\mactag}{\algo{Tag}}
\newcommand{\macver}{\algo{Ver}}

% key-encapsulation mechanism
\newcommand{\kem}{\scheme{KEM}}
\newcommand{\kemgen}{\algo{Gen}}
\newcommand{\kemenc}{\algo{Encaps}}
\newcommand{\kemdec}{\algo{Decaps}}

% identity-based encryption
\newcommand{\ibe}{\scheme{IBE}}
\newcommand{\ibesetup}{\algo{Setup}}
\newcommand{\ibeext}{\algo{Ext}}
\newcommand{\ibeenc}{\algo{Enc}}
\newcommand{\ibedec}{\algo{Dec}}

% hierarchical IBE (as key encapsulation)
\newcommand{\hibe}{\scheme{HIBE}}
\newcommand{\hibesetup}{\algo{Setup}}
\newcommand{\hibeext}{\algo{Extract}}
\newcommand{\hibeenc}{\algo{Encaps}}
\newcommand{\hibedec}{\algo{Decaps}}

% binary tree encryption (as key encapsulation)
\newcommand{\bte}{\scheme{BTE}}
\newcommand{\btesetup}{\algo{Setup}}
\newcommand{\bteext}{\algo{Extract}}
\newcommand{\bteenc}{\algo{Encaps}}
\newcommand{\btedec}{\algo{Decaps}}

% trapdoor functions
\newcommand{\tdf}{\scheme{TDF}}
\newcommand{\tdfgen}{\algo{Gen}}
\newcommand{\tdfeval}{\algo{Eval}}
\newcommand{\tdfinv}{\algo{Invert}}
\newcommand{\tdfver}{\algo{Ver}}

%%% PROTOCOLS

\newcommand{\out}{\text{out}}
\newcommand{\view}{\text{view}}

%%% COMMANDS FOR LECTURES/HOMEWORKS

\newcommand{\lecheader}{%
  \chead{\large \textbf{Lecture \lecturenum\\\lecturetopic}}

  \lhead{\small \textbf{Theory of Cryptography}\\}

  \rhead{\small \textbf{Instructor:
      \href{http://www.eecs.umich.edu/~cpeikert/}{Chris Peikert}\\Scribe:
      \scribename}}

  \setlength{\headheight}{20pt}
  \setlength{\headsep}{16pt}
}


% VARIABLES

\newcommand{\lecturenum}{15}
\newcommand{\lecturetopic}{Zero Knowledge, Interactive Proofs}
\newcommand{\scribename}{Abhinav Shantanam}

% END OF VARIABLES

\lecheader

\pagestyle{plain}               % default: no special header

\begin{document}

\thispagestyle{fancy} % first page should have special header

% LECTURE MATERIAL STARTS HERE

\section{Zero Knowledge}
\label{sec:zero-knowledge}

\subsection{Warm-Up: Zero Information}
\label{sec:warm-up-zero-inf}

When does a random variable $Y$ reveal \emph{no information} about
another random variable $X$?  In probability theory, the answer to
this question is: when $X$ and $Y$ are \emph{independent}.  One
equivalent form of independence is that for \emph{any} fixed value of
$X$, the conditional distribution of $Y$ (given $X$) is always just
some certain distribution $D_{Y}$.

Observe that perfect secrecy can be seen in this light: a shared-key
encryption scheme is perfectly secure if (and only if) for any message
$m \in \msgspace$, the distribution of the ciphertext $c \in \ctspace$
(over the choice of $k \gets \skcgen$) is the same: some fixed
distribution $D'$ over $\ctspace$.

\subsection{Zero \emph{Computational} Knowledge}
\label{sec:zero-comp-knowl}

What is the \emph{computational} analogue of revealing no information?
When working with bounded computation, we speak of \emph{knowledge}
instead of information.  This is because even though a value $Y$ might
reveal \emph{all} the information in $X$, that information might not
be accessible to a bounded algorithm.  For example, for a one-way
permutation $f$ with hard-core predicate $h$, the value $f(x)$ reveals
all the information in $x$, but it gives no \emph{knowledge} about
$h(x)$, because $h(x)$ ``looks like'' a uniformly random bit.  Seen
another way: up to computational indistinguishability, $h(x)$ ``might
as well have been'' produced by an \emph{efficient algorithm that is
  not given $x$ at all!}  (That is, the algorithm that just outputs a
uniformly random bit.)

Let us see a concrete example of this kind of definition for
shared-key encryption.  Informally, it says that a ciphertext of a
message $m$ might as well be produced by an efficient algorithm (a
simulator) that is not given the message.  (Note the syntactic
similarity to the information-theoretic statements above.)

\begin{definition}
  \label{def:zk-skc}
  A shared-key cryptosystem $\skc$ for message space $\msgspace$ is
  \emph{zero-knowledge} if there exists an nuppt \emph{simulator}
  $\Sim$ such that for all $m \in \msgspace$, \[ \set{k \gets
    \skcgen(1^{n}) : \skcenc_{k}(m)} \compind \set{\Sim(1^{n})}. \]
\end{definition}

Recall our simplest definition of security for shared-key encryption,
one-time indistinguishability, which says that for all $m_{0}, m_{1}
\in \msgspace$, \[ \set{k \gets \skcgen : \skcenc_{k}(m_{0})} \compind
\set{k \gets \skcgen : \skcenc_{k}(m_{1})}. \]

\begin{theorem}
  \label{thm:zk-iff-indist}
  A shared-key cryptosystem $\skc$ is one-time indistinguishable if
  and only if it is zero-knowledge.
\end{theorem}

\begin{proof}
  If $\skc$ is zero-knowledge, let $\Sim$ be the simulator guaranteed
  by Definition~\ref{def:zk-skc}.  Then by hypothesis, for all $m_{0},
  m_{1} \in \mathcal{M}$,
  \[ \set{k \gets \skcgen(1^{n}): \skcenc_{k}(m_0)} \compind
  \set{\Sim(1^{n})} \compind \set{k \gets \skcgen(1^{n}) :
    \skcenc_{k}(m_1)}. \] One-time indistinguishability follows by the
  hybrid lemma.

  If $\skc$ is one-time indistinguishable, let $m' \in \msgspace$
  denote some arbitrary message, and define $\Sim(1^{n})$ to choose $k
  \gets \skcgen(1^{n})$ and output $c \gets \skcenc_{k}(m')$.  Then by
  one-time indistinguishability, for any $m \in \msgspace$ we have
  \[ \set{k \gets \skcgen(1^{n}): \skcenc_{k}(m)} \compind \set{k
    \gets \skcgen(1^{n}) : \skcenc_{k}(m')} \equiv
  \set{\Sim(1^{n})}. \qedhere \]
\end{proof}

\subsection{Why Zero Knowledge?}
\label{sec:why-zero-knowledge}

What do we gain from zero-knowledge/simulation-style security
definitions?  So far, not much other than a new perspective.  However,
this new perspective is very powerful: it can capture, in a unified
way, many of the disparate security definitions we have seen so far
(e.g., for pseudorandom generators/functions, signatures, etc.).  Even
more importantly, this perspective will allow us to give meaningful
security definitions for protocols in which the adversary may be an
\emph{active participant}.  By ``active'' we mean not just that the
adversary can make queries to the scheme (e.g., chosen-message or
chosen-ciphertext attacks), but that it instead \emph{plays the role}
of one of the scheme's algorithms (possibly in a malicious way).

\section{Interactive Proofs}
\label{sec:interactive-proofs}

\subsection{Motivation}
\label{sec:motivation}

What is a (mathematical) proof?  Two of the most common answers
include:
\begin{enumerate}
\item A sequence of sound logical deductions, starting from a set of
  axioms and/or hypotheses, that concludes in a theorem?
\item An argument that convinces the peer community?
\end{enumerate}

A ``classical'' proof (in the mold of the first answer above) is a
\emph{static} object, which can be easily checked by applying
mechanical rules.  In contrast, many proofs in mathematics have
required a lot of effort to find --- much more effort than it takes to
verify them.  Intuitively, then, a classical proof may reveal more
than zero knowledge to the party verifying it, in that the verifier
may not have been able to generate the proof itself.

In practice, many proofs are partially ``interactive:'' a prover
asserts a theorem, and a verifier (e.g., the peer community) asks
questions of the prover until it is satisfied that the theorem is
correct.  It turns out that this style of proof in some cases appears
to be qualitatively \emph{more powerful} than a static, classical
proof.

Consider the following example ``theorem:'' Coke and Pepsi are
different soft drinks.  To prove this claim, it might not suffice to
give different recipes for the two drinks.  First, how does one check
that a given recipe corresponds to the claimed drink?  And even if one
could, how can one be sure that two different recipes don't end up
producing the same end product (e.g., by a different sequence of
chemical reactions)?

On the other hand, consider the following interactive ``proof'' for
establishing the theorem: the verifier secretly buys either a bottle
of Coke or Pepsi at random, pours it into a glass, and asks the prover
to identify the drink (by looking at, tasting it, or whatever the
prover requires to decide).  The prover and verifier may repeat the
exercise multiple times (with a fresh random choice of drink each
time).  If the theorem is indeed true (Coke and Pepsi are different),
then a sufficiently talented prover can always identify correctly.
However, if the theorem is false (Coke and Pepsi are exactly the
same), then the prover would have only a $1/2$ chance of guessing
successfully in each iteration.  Therefore, if the prover succeeds in
every iteration, the verifier can conclude --- with very high but not
complete confidence --- that the theorem is true.

\subsection{Formalization}
\label{sec:formalization}

Let $P$ and $V$ denote ``interactive'' algorithms, which take inputs
and random coins as usual, but also (informally speaking) can pass
messages to each other.  Only one machine at a time performs ``work;''
when it passes a message to the other, the other machine starts
working.  We denote their joint execution on their respective inputs
by $P(\cdot) \leftrightarrow V(\cdot)$.  The notation
$\out_{V}[\cdot]$ denotes the output of $V$ in the specified
interaction.  A \emph{language} $L$ is just a subset of $\bit^{*}$,
whose elements we often call ``theorems.''

\begin{definition}
  \label{def:ips}
  An \emph{interactive proof system} (with \emph{soundness error} $s
  \in [0,1]$) for a language $L \subseteq \bit^{*}$ is a pair of
  algorithms: a (possibly computationally unbounded) prover $P$, and a
  ppt verifier $V$, with the following properties:
  \begin{enumerate}
  \item \emph{Completeness} (``the specified prover convinces the
    specified verifier of a true statement''): For all $x \in L$,
    $\out_{V}[ P(x) \leftrightarrow V(x) ] = 1$, with probability $1$.
  \item \emph{Soundness} (``\emph{no} possibly cheating prover can
    convince the specified verifier of a false statement''): For every
    computationally unbounded $P^*$ and all $x \not\in L$,
    \[ \Pr\left[ \out_{V}[ P^{*}(x) \leftrightarrow V(x) ] = 1 \right]
    \leq s. \]
  \end{enumerate}
\end{definition}

\begin{remark}
  \label{rem:ips}
  Some notes on the above definition:
  \begin{itemize}
  \item Completeness is a property of the specified (``honest'')
    prover and verifier algorithms, and says nothing about security.
    Soundness protects the verifier: as long as it runs its specified
    $V$ algorithm, a cheating prover --- \emph{no matter how it might
      arbitrarily deviate from the protocol!} --- will not be able to
    convince it of a false theorem (with probability greater than
    $s$).
  \item We could relax the completeness condition to require
    $\out_{V}[ P(x) \leftrightarrow V(x) ] = 1$ with probability at
    least $1-c > s$ for some $c$ called the \emph{completeness} error.
    Every proof system we'll see in this course will have
    \emph{perfect} completeness, i.e., $c=0$.
  \item If we allow the probabilistic verifier machine and the
    all-powerful prover to interact for a polynomial number of rounds,
    we get the class of problems called IP.  In 1992, Adi Shamir found
    one of the central results of complexity theory: IP equals PSPACE,
    the class of problems solvable by an ordinary deterministic Turing
    machine using polynomial space (and unbounded time).
  \end{itemize}
\end{remark}

\subsection{Interactive Proof for Graph Nonisomorphism}
\label{sec:inter-proof-graph}

A formal analogue of the Coke-versus-Pepsi question is the \emph{graph
nonisomorphism} problem, defined below.

\newcommand{\gni}{\problem{GNI}}

A graph $G = (V,E)$ consists of a finite set $V$ of vertices, and an
edge set $E \subseteq V \times V$ (where $(v_{0}, v_{1}) \in E$ means
that there is an edge between vertices $v_{0}$ and $v_{1}$).  Without
loss of generality, we often say that the vertex set is $[n] = \set{1,
\ldots, n}$ for some nonnegative integer $n$.

\begin{definition}
  Two graphs $G_{0} = (V_{0}, E_{0})$ and $G_{1} = (V_{1}, E_{1})$ are
  \emph{isomorphic}, written $G_{0} \equiv G_{1}$, if there exists a
  bijection $\rho \colon V_{0} \to V_{1}$ such that $(v_{0}, v_{1})
  \in E_{0}$ if and only if $(\rho(v_{0}), \rho(v_{1})) \in E_{1}$.
\end{definition}

\begin{definition}
  \label{def:gni}
  The \emph{graph nonisomorphism language} $\gni$ is defined as \[
  \gni = \set{(G_{0}, G_{1}) : G_{0}, G_{1} \text{ are graphs such
      that } G_{0} \not\equiv G_{1}}. \]
\end{definition}

To date, it is unknown whether there exists an efficient algorithm
that decides the graph (non)isomorphism language (though there
\emph{are} efficient algorithms that decide it for a broad class of
special cases).  It is not even known if $\gni$ is in $\NP$, i.e., if
membership in $\gni$ can be efficiently verified given a
(polynomial-length) static proof.  However, we are able to give a very
simple \emph{interactive} proof for $\gni$.

\begin{protocol}
  We define the following verifier and prover algorithms for $\gni$,
  which interact as follows:
  \begin{enumerate}
  \item $V(G_{0}, G_{1})$: Choose $b \gets \bit$ and a uniformly
    random permutation $\pi$ of $V_{b}$ (the vertex set of $G_{b}$).
    Send the permuted graph $H = \rho(G_{b})$ to the prover.
    (Intuitively, the verifier is challenging the prover to identify
    which of the original two graphs $H$ is isomorphic to.)
  \item $P(G_{0}, G_{1})$: Upon receiving a graph $H$ from the
    verifier, find (perhaps using a large amount of computation) a $b'
    \in \bit$ such that $G'$ is isomorphic to $G_{b'}$.  Send $b'$ to
    the verifier.  (Note that if $G_{0}$ and $G_{1}$ are not
    isomorphic, then there is at most one valid $b'$.)
  \item $V(G_{0}, G_{1})$: If $b' = b$, accept; otherwise, reject.
  \end{enumerate}
\end{protocol}

\begin{theorem}
  The algorithm pair $(P,V)$ described above forms an interactive
  proof system for $\gni$, with soundness error $1/2$.
\end{theorem}

\begin{proof}
  First we show completeness.  Let $(G_{0}, G_{1}) \in \gni$, i.e.,
  $G_0$ and $G_1$ are not isomorphic.  Then no graph can be isomorphic
  to both $G_0$ and $G_1$.  It follows that for any $H$ generated by
  (honest) $V$ with random choice $b$, the bit $b'$ computed by the
  prover always equals the $b$ chosen by the verifier, and $V$ always
  accepts.
  
  Now we show soundness.  If $(G_0, G_{1}) \not\in \gni$, i.e.,
  $G_{0}$ and $G_1$ are isomorphic, then a uniformly random
  permutation of $G_{0}$ and $G_{1}$ are identically distributed.
  More precisely, suppose that $G_{1} = \pi(G_{0})$ for some
  permutation $\pi$.  Then for any permutation $\rho$, we have
  $\rho(G_{1}) = (\rho \circ \pi)(G_{0})$.  In particular, if $\rho$
  is a uniformly random permutation, then $\rho \circ \pi$ is also
  uniformly random, hence $\rho(G_{0})$ and $(\rho \circ \pi)(G_{0})$
  are identically distributed.  Therefore, the graph $H$ sent by the
  verifier is statistically independent of its internal bit $b$, and
  for any prover reply $b'$, the probability that $b' = b$ is exactly
  $1/2$, as desired.
\end{proof}

\end{document}

%%% Local Variables: 
%%% mode: latex
%%% TeX-master: t
%%% End: 
