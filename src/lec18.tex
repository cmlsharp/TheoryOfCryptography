\documentclass[11pt]{article}

\usepackage{fullpage}
\usepackage{newtxtext}
\usepackage[T1]{fontenc}
\usepackage{hyperref,microtype,pdfsync}
\usepackage{amsmath,amsfonts,amssymb,amsthm}
\usepackage{mathtools}
\usepackage{fancyhdr}

%%% BLACKBOARD SYMBOLS

% hyperref package defines C and G, undefined to avoid conflicts
\let\C\relax
\let\G\relax
\newcommand{\C}{\ensuremath{\mathbb{C}}}
\newcommand{\D}{\ensuremath{\mathbb{D}}}
\newcommand{\F}{\ensuremath{\mathbb{F}}}
\newcommand{\G}{\ensuremath{\mathbb{G}}}
\newcommand{\J}{\ensuremath{\mathbb{J}}}
\newcommand{\N}{\ensuremath{\mathbb{N}}}
\newcommand{\Q}{\ensuremath{\mathbb{Q}}}
\newcommand{\R}{\ensuremath{\mathbb{R}}}
\newcommand{\T}{\ensuremath{\mathbb{T}}}
\newcommand{\Z}{\ensuremath{\mathbb{Z}}}
\newcommand{\QR}{\ensuremath{\mathbb{QR}}}

\newcommand{\Zt}{\ensuremath{\Z_t}}
\newcommand{\Zp}{\ensuremath{\Z_p}}
\newcommand{\Zq}{\ensuremath{\Z_q}}
\newcommand{\ZN}{\ensuremath{\Z_N}}
\newcommand{\Zps}{\ensuremath{\Z_p^*}}
\newcommand{\ZNs}{\ensuremath{\Z_N^*}}
\newcommand{\JN}{\ensuremath{\J_N}}
\newcommand{\QRN}{\ensuremath{\QR_{N}}}
\newcommand{\QRp}{\ensuremath{\QR_{p}}}

%%% THEOREM COMMANDS

\theoremstyle{plain}            % following are "theorem" style
\newtheorem{theorem}{Theorem}[section]
\newtheorem{lemma}[theorem]{Lemma}
\newtheorem{corollary}[theorem]{Corollary}
\newtheorem{proposition}[theorem]{Proposition}
\newtheorem{claim}[theorem]{Claim}
\newtheorem{fact}[theorem]{Fact}

\theoremstyle{definition}       % following are def style
\newtheorem{definition}[theorem]{Definition}
\newtheorem{conjecture}[theorem]{Conjecture}
\newtheorem{example}[theorem]{Example}
\newtheorem{protocol}[theorem]{Protocol}

\theoremstyle{remark}           % following are remark style
\newtheorem{remark}[theorem]{Remark}
\newtheorem{note}[theorem]{Note}
\newtheorem{exercise}[theorem]{Exercise}

% equation numbering style
\numberwithin{equation}{section}

%%% GENERAL COMPUTING

\newcommand{\bit}{\ensuremath{\set{0,1}}}
\newcommand{\pmone}{\ensuremath{\set{-1,1}}}

% asymptotics
\DeclareMathOperator{\poly}{poly}
\DeclareMathOperator{\polylog}{polylog}
\DeclareMathOperator{\negl}{negl}
\newcommand{\Otil}{\ensuremath{\tilde{O}}}

% probability/distribution stuff
\DeclareMathOperator*{\E}{E}
\DeclareMathOperator*{\Var}{Var}

% sets in calligraphic type
\newcommand{\calD}{\ensuremath{\mathcal{D}}}
\newcommand{\calF}{\ensuremath{\mathcal{F}}}
\newcommand{\calG}{\ensuremath{\mathcal{G}}}
\newcommand{\calH}{\ensuremath{\mathcal{H}}}
\newcommand{\calX}{\ensuremath{\mathcal{X}}}
\newcommand{\calY}{\ensuremath{\mathcal{Y}}}

% types of indistinguishability
\newcommand{\compind}{\ensuremath{\stackrel{c}{\approx}}}
\newcommand{\statind}{\ensuremath{\stackrel{s}{\approx}}}
\newcommand{\perfind}{\ensuremath{\equiv}}

% font for general-purpose algorithms
\newcommand{\algo}[1]{\ensuremath{\mathsf{#1}}}
% font for general-purpose computational problems
\newcommand{\problem}[1]{\ensuremath{\mathsf{#1}}}
% font for complexity classes
\newcommand{\class}[1]{\ensuremath{\mathsf{#1}}}

% complexity classes and languages
\renewcommand{\P}{\class{P}}
\newcommand{\BPP}{\class{BPP}}
\newcommand{\NP}{\class{NP}}
\newcommand{\coNP}{\class{coNP}}
\newcommand{\AM}{\class{AM}}
\newcommand{\coAM}{\class{coAM}}
\newcommand{\IP}{\class{IP}}

%%% "LEFT-RIGHT" PAIRS OF SYMBOLS

%% NOTE: this requires \usepackage{mathtools} in the document preamble

% inner product
\DeclarePairedDelimiter\inner{\langle}{\rangle}
% absolute value
\DeclarePairedDelimiter\abs{\lvert}{\rvert}
% a set
\DeclarePairedDelimiter\set{\{}{\}}
% parens
\DeclarePairedDelimiter\parens{(}{)}
% tuple, alias for parens
\DeclarePairedDelimiter\tuple{(}{)}
% square brackets
\DeclarePairedDelimiter\bracks{[}{]}
% rounding off
\DeclarePairedDelimiter\round{\lfloor}{\rceil}
% floor function
\DeclarePairedDelimiter\floor{\lfloor}{\rfloor}
% ceiling function
\DeclarePairedDelimiter\ceil{\lceil}{\rceil}
% length of some vector, element
\DeclarePairedDelimiter\length{\lVert}{\rVert}
% "lifting" of a residue class
\DeclarePairedDelimiter\lift{\llbracket}{\rrbracket}
\DeclarePairedDelimiter\len{\lvert}{\rvert}

%%% CRYPTO-RELATED NOTATION

% KEYS AND RELATED

\newcommand{\key}[1]{\ensuremath{#1}}

\newcommand{\pk}{\key{pk}}
\newcommand{\vk}{\key{vk}}
\newcommand{\sk}{\key{sk}}
\newcommand{\mpk}{\key{mpk}}
\newcommand{\msk}{\key{msk}}
\newcommand{\fk}{\key{fk}}
\newcommand{\id}{id}
\newcommand{\keyspace}{\ensuremath{\mathcal{K}}}
\newcommand{\msgspace}{\ensuremath{\mathcal{M}}}
\newcommand{\ctspace}{\ensuremath{\mathcal{C}}}
\newcommand{\tagspace}{\ensuremath{\mathcal{T}}}
\newcommand{\idspace}{\ensuremath{\mathcal{ID}}}

\newcommand{\concat}{\ensuremath{\|}}

% GAMES

% advantage
\newcommand{\advan}{\ensuremath{\mathbf{Adv}}}

% different attack models
\newcommand{\attack}[1]{\ensuremath{\text{#1}}}

\newcommand{\atk}{\attack{atk}} % dummy attack
\newcommand{\indcpa}{\attack{ind-cpa}}
\newcommand{\indcca}{\attack{ind-cca}}
\newcommand{\anocpa}{\attack{ano-cpa}} % anonymous
\newcommand{\anocca}{\attack{ano-cca}}
\newcommand{\euacma}{\attack{eu-acma}} % forgery: adaptive chosen-message
\newcommand{\euscma}{\attack{eu-scma}} % forgery: static chosen-message
\newcommand{\suacma}{\attack{su-acma}} % strongly unforgeable

% ADVERSARIES
\newcommand{\attacker}[1]{\ensuremath{\mathcal{#1}}}

\newcommand{\Adv}{\attacker{A}}
\newcommand{\AdvA}{\attacker{A}}
\newcommand{\AdvB}{\attacker{B}}
\newcommand{\Dist}{\attacker{D}}
\newcommand{\Sim}{\attacker{S}}
\newcommand{\Ora}{\attacker{O}}
\newcommand{\Inv}{\attacker{I}}
\newcommand{\For}{\attacker{F}}

% CRYPTO SCHEMES

\newcommand{\scheme}[1]{\ensuremath{\text{#1}}}

% pseudorandom stuff
\newcommand{\prg}{\algo{PRG}}
\newcommand{\prf}{\algo{PRF}}
\newcommand{\prp}{\algo{PRP}}

% symmetric-key cryptosystem
\newcommand{\skc}{\scheme{SKC}}
\newcommand{\skcgen}{\algo{Gen}}
\newcommand{\skcenc}{\algo{Enc}}
\newcommand{\skcdec}{\algo{Dec}}

% public-key cryptosystem
\newcommand{\pkc}{\scheme{PKC}}
\newcommand{\pkcgen}{\algo{Gen}}
\newcommand{\pkcenc}{\algo{Enc}} % can also use \kemenc and \kemdec
\newcommand{\pkcdec}{\algo{Dec}}

% digital signatures
\newcommand{\sig}{\scheme{SIG}}
\newcommand{\siggen}{\algo{Gen}}
\newcommand{\sigsign}{\algo{Sign}}
\newcommand{\sigver}{\algo{Ver}}

% message authentication code
\newcommand{\mac}{\scheme{MAC}}
\newcommand{\macgen}{\algo{Gen}}
\newcommand{\mactag}{\algo{Tag}}
\newcommand{\macver}{\algo{Ver}}

% key-encapsulation mechanism
\newcommand{\kem}{\scheme{KEM}}
\newcommand{\kemgen}{\algo{Gen}}
\newcommand{\kemenc}{\algo{Encaps}}
\newcommand{\kemdec}{\algo{Decaps}}

% identity-based encryption
\newcommand{\ibe}{\scheme{IBE}}
\newcommand{\ibesetup}{\algo{Setup}}
\newcommand{\ibeext}{\algo{Ext}}
\newcommand{\ibeenc}{\algo{Enc}}
\newcommand{\ibedec}{\algo{Dec}}

% hierarchical IBE (as key encapsulation)
\newcommand{\hibe}{\scheme{HIBE}}
\newcommand{\hibesetup}{\algo{Setup}}
\newcommand{\hibeext}{\algo{Extract}}
\newcommand{\hibeenc}{\algo{Encaps}}
\newcommand{\hibedec}{\algo{Decaps}}

% binary tree encryption (as key encapsulation)
\newcommand{\bte}{\scheme{BTE}}
\newcommand{\btesetup}{\algo{Setup}}
\newcommand{\bteext}{\algo{Extract}}
\newcommand{\bteenc}{\algo{Encaps}}
\newcommand{\btedec}{\algo{Decaps}}

% trapdoor functions
\newcommand{\tdf}{\scheme{TDF}}
\newcommand{\tdfgen}{\algo{Gen}}
\newcommand{\tdfeval}{\algo{Eval}}
\newcommand{\tdfinv}{\algo{Invert}}
\newcommand{\tdfver}{\algo{Ver}}

%%% PROTOCOLS

\newcommand{\out}{\text{out}}
\newcommand{\view}{\text{view}}

%%% COMMANDS FOR LECTURES/HOMEWORKS

\newcommand{\lecheader}{%
  \chead{\large \textbf{Lecture \lecturenum\\\lecturetopic}}

  \lhead{\small \textbf{Theory of Cryptography}\\}

  \rhead{\small \textbf{Instructor:
      \href{http://www.eecs.umich.edu/~cpeikert/}{Chris Peikert}\\Scribe:
      \scribename}}

  \setlength{\headheight}{20pt}
  \setlength{\headsep}{16pt}
}


% VARIABLES

\newcommand{\lecturenum}{18}
\newcommand{\lecturetopic}{Proofs of Knowledge}
\newcommand{\scribename}{Joel Odom}

% END OF VARIABLES

\lecheader

\pagestyle{plain}              % default: no special header

\begin{document}

\thispagestyle{fancy}          % first page should have special header

% LECTURE MATERIAL STARTS HERE

\section{Proofs of Knowledge}
\label{sec:pok}

Up to this point we have studied zero-knowledge proofs that are
capable of proving whether a statement is ``true'' or ``false,'' i.e.,
whether a statement lies in or out of a language.  Sometimes, though,
the notion of membership in a language does not capture the claim that
the prover wants to demonstrate.  For example, if the statement is
true, might it be that \emph{anyone} can give an acceptable proof?
The definition of an interactive proof doesn't give us much insight
into this question; it only says that \emph{false} theorems can't be
proved (even by an unbounded prover).  Here we will see that it is
sometimes useful to have a \emph{proof of knowledge}, in which a
prover convinces a verifier that the prover ``knows why'' a statement
is true.

Take, for example, the application we considered at the end of the
last lecture.  Alice wanted to convince Bob, in zero knowledge, that
she is a CIA agent --- or more precisely, that the CIA has signed the
message ``Alice is a CIA agent'' under its public key.  (ZK is useful
here because it would make the proof nontransferable by Bob.)  How can
we go about proving this in ZK?  If Alice just reveals the signature,
that conveys knowledge to Bob that he could not have simulated himself
(and is transferable).  If Alice proves merely that there
\emph{exists} a valid signature (under the CIA's public key) of the
message ``Alice is a CIA agent,'' this doesn't necessarily prove
anything meaningful, because by the completeness of the signature
scheme, there exists a signature for \emph{every} message!  What Alice
really wants to prove is that she \emph{knows} a valid signature of
the appropriate message.

\subsection{Example Protocol}
\label{sec:example-protocol}

What we want to capture is not only the truth of a proposition ($x \in
L$), but also that the prover ``possesses'' a suitable witness to this
fact.  We demonstrate this concept by an example protocol which proves
knowledge of a discrete logarithm in a cyclic group.

\begin{protocol}
  \label{prot:dlog}
  Let $G = \langle g \rangle$ be a cyclic group of known order $q$
  with known generator $g$, and let $x \in G$ be an arbitrary group
  element having discrete logarithm $w = \log_{g}(x)$.  The common
  input to $P$ and $V$ is $x$, and $P$ is additionally given $w$.
  \begin{enumerate}

  \item $P(x,w)$ selects a random $r \gets \Zq$ and sends $z = g^{r}$
    to $V$.

  \item $V(x)$ sends a random challenge bit $b \in \bit$ to $P$.  (Here
    $V$ is challenging the prover to reply with the discrete log of $z$,
    or of $z \cdot x$.)

  \item $P$ responds with $a = r + b \cdot w \in \Zq$.
    
  \item $V$ accepts if $g^{a} = z \cdot x^{b}$.

  \end{enumerate}
  Schematically,

  \begin{center}
    \begin{tabular}{ccc}
      $P(x,w = \log_{g}(y))$ & & $V(x)$ \\ \\
      $r \gets \Zq$ & $\underrightarrow{\quad z = g^{r} \quad}$ & \\
      & & $b \gets \bit$ \\
      & $\underleftarrow{\quad b \quad}$ & \\
      & & \\
      & $\underrightarrow{\quad a = r + b w \quad}$ & accept iff
      $g^{a}=z\cdot x^{b}$
    \end{tabular}
  \end{center}
\end{protocol}

Is the protocol \emph{complete}?  Yes: if $P$ and $V$ act as
described, then $g^{a} = g^{r + bw} = g^{r} \cdot (g^{w})^{b} = z
\cdot x^{b}$.  Is the protocol \emph{zero-knowledge}?  Let $V^{*}$ be
a (possibly malicious) nuppt verifier.  We define a simulator
$\Sim^{V^{*}}(x)$:
\begin{enumerate}

\item Choose a random bit $b \gets \bit$ and a random group element $a
  \gets G$.  (This can be done by choosing $t \gets \Zq$ and letting
  $a = g^{t}$.)

\item Send $z = g^{a} / x^{b}$ to $V^{*}$ (run with fresh random
  coins) and get back a challenge bit $b^{*}$.  (If $V^{*}$ responds
  with a malformed message or aborts, just output the view so far.)

\item If $b^{*} = b$, complete the view using $a$ as the prover's last
  message, otherwise rewind $V^{*}$ and repeat the simulation (up to
  $n$ total iterations).
\end{enumerate}
It is relatively easy to show that $\Sim$ reproduces $V^{*}$'s view,
up to negligible statistical distance.  (Briefly, this is because the
value $z$ calculated by $\Sim$ is statistically independent of its bit
$b$, so $b^{*} = b$ with probability $1/2$.)

Is the protocol \emph{sound}?  Note that \emph{every} $x \in G$ has a
discrete logarithm, so the notion of soundness is not particularly
meaningful here!  A separate question is whether the protocol gives
evidence that the prover possesses \emph{knowledge} of the discrete
logarithm $x$.  An informal inspection of the protocol seems to
indicate that it does: if the prover can answer both challenges, then
it is prepared to reply with both $\log_{g}(z)$ and $\log_{g}(z \cdot
x) = \log_{g}(z) + \log_{g}(x)$, whose difference is exactly
$\log_{g}(x)$.  But we need to formalize some notions before we can
come to any definite conclusion.

\subsection{Definition of an $\NP$-Relation}
\label{sec:def-pok}

\begin{definition}
  \label{def:NP-relation}
  An \emph{$\NP$-relation} $R \subseteq \bit^{*} \times \bit^{*}$ is
  given by a deterministic algorithm $W(\cdot,\cdot)$ that runs in
  time polynomial in the length of its first input.  The relation
  is \[ R = \set{(x,w) : W(x,w) \text{ accepts}}. \] The associated
  $\NP$-language $L_{R} = \set{x : \exists\; w \text{ such that }
    W(x,w) \text{ accepts}}$.  The witness set for an $x \in \bit^{*}$
  is $R(x) = \set{w : W(x,w) = 1}$.
\end{definition}

That is, the $\NP$-relation $R$ is just the set of theorem-witness
pairs $(x,w)$ that are accepted by the verifier $W$.  It is easily
seen that the language $L_{R}$ associated with the relation is an
$\NP$ language.

\begin{example}
  \label{ex:dlog-relation}
  For the discrete logarithm problem in $G$, the natural relation
  is \[ R = \set{(x \in G, w \in \Zq) : g^{w} = x}. \]
\end{example}

\subsection{Definition of a Proof of Knowledge}
\label{sec:defin-proof-knowl}

We can now define a proof of knowledge for a relation $R$.  The
definition captures the intuition that from \emph{any} (possibly
cheating) prover $P^{*}$ that is able to convince the verifier with
good enough probability on a statement $x \in L_{R}$, there is a way
to extract a valid witness for $x$ from $P^{*}$ with a related (not
too small) probability.

\begin{definition}
  \label{def:pok}
  An interactive proof system $(P,V)$ is a \emph{proof of knowledge}
  with \emph{knowledge error} $\kappa \in [0,1]$ for an $\NP$-relation
  $R$ if there exists an nuppt oracle machine $K^{\cdot}$ (the
  \emph{knowledge extractor}) such that for any $x \in L_{R}$, and for
  any (possibly unbounded) $P^{*}$ for which $p^{*}_{x} =
  \Pr[\out_{V}[P^{*} \leftrightarrow V(x)] = 1] > \kappa$, we have \[
  \Pr[K^{P^{*}}(x) \in R(x)] \geq \poly(p^{*}_{x} - \kappa). \]
\end{definition}

In words, the probability that the knowledge extractor $K$ finds a
valid witness for $x$ using its access to prover $P^{*}$ is at least
polynomially related to the probability $p^{*}_{x}$ that $P^{*}$
convinces the honest verifier on $x$, less some knowledge error.

\begin{remark}
  \label{rem:pok}
  A number of remarks on this definition are appropriate.

  \begin{enumerate}

  \item We don't make any restriction on the kind of prover $P^{*}$
    under consideration: it could be unbounded and/or malicious.

  \item The definition is \emph{per instance} $x$: if $P^{*}$
    convinces $V$ that a \emph{particular} $x$ is true, then it should
    know a witness for \emph{that} $x$.

  \item Notice that in the discrete logarithm protocol above, there is
    a trivial cheating prover that convinces the verifier with
    probability $\frac{1}{2}$.  (This is the prover that prepares an
    answer for a known $b$ and succeeds only if the verifier issues
    $b$ as its challenge bit.)  There is no hope of extracting any
    witness from this prover, without just solving the discrete log
    problem directly.  Just as with soundness error, the knowledge
    error $\kappa$ captures the ``threshold'' probability with which a
    cheating prover $P^{*}$ must convince the verifier before it can
    be said to know a valid witness.

  \item Because the relation $R$ is efficiently checkable, we can run
    $K^{P^{*}}$ many times and output any valid witness for $x$ that
    is found in any execution.  The expected running time of this is
    $1/\poly(p^{*}_{x} - \kappa)$.  If $p^{*}_{x}-\kappa$ is at least
    inverse-polynomial in the security parameter, then a witness is
    extracted in expected polynomial time.

  \item The definition is not formally related to soundness (i.e.,
    neither property necessarily implies the other).  However, notice
    that the definition only applies to $x \in L_{R}$, because
    otherwise the set $R(x)$ is empty and therefore $K$ could never
    output a valid witness.
  \end{enumerate}
\end{remark}

\subsection{Knowledge Extractor for Discrete Log Protocol}
\label{sec:knowl-extr-discr}

\begin{theorem}
  Protocol~\ref{prot:dlog} is a proof of knowledge for the discrete
  logarithm relation, with knowledge error $\kappa = 1/2$.
\end{theorem}

\begin{proof}
  We construct a knowledge extractor $K^{P^{*}}(x)$ that works as
  follows:
  \begin{enumerate}
  \item Run $P^{*}$ and receive a message $z \in G$ from $P^{*}$.
  \item Send challenge bit $b = 0$ to $P^{*}$ and receive an answer
    $a_{0}$.\label{item:chall-zero}
  \item Rewind $P^{*}$ to its state before Step~\ref{item:chall-zero}.
    Send challenge bit $b = 1$ to $P^{*}$, and receive an answer
    $a_{1}$.
  \item Output $a_{1} - a_{0}$.
  \end{enumerate}

  Note that \emph{if} $a_{0}$ and $a_{1}$ are both correct answers to
  the respective challenges on $z$, then $a_{1} - a_{0} = \log_{g}(z
  \cdot x) - \log_{g}(z) = \log_{g}(x)$, so the knowledge extractor's
  output is a correct witness for $x$.

  Suppose that $\Pr[\out_{V}[P^{*} \leftrightarrow V(x)] = 1] = 1/2 +
  \alpha$ for some positive $\alpha$ (otherwise, we have nothing to
  prove).  We claim that \[ \Pr[a_{0} \text{ and } a_{1} \text{
    correct}] \geq \alpha^{3}/2 = \poly(\alpha), \] as required by
  Definition~\ref{def:pok}.

  To prove the claim, define \[ p_{z} = \Pr[P^{*} \text{ convinces }
  V(x) \mid P^{*} \text{ outputs $z$ in the first message}]. \] By an
  averaging argument (Markov inequality), we have $\Pr_{z \gets
    P^{*}}[p_{z} \geq \tfrac{1}{2} + \tfrac{\alpha}{2}] \geq
  \frac{\alpha}{2}$.

  Now consider any such ``good'' $z$, where $p_{z} \geq \tfrac12 +
  \tfrac{\alpha}{2}$.  Define \[ p_{z, b} = \Pr[P^{*} \text{ convinces
  } V(x) \mid z \text{, challenge $b$}]. \] Because $p_{z} = (p_{z,0}
  + p_{z,1})/2$ and both $p_{z,0}, p_{z,1} \leq 1$, we have both
  $p_{z, 0}, p_{z, 1} \geq \alpha$.  The events that $P^{*}$ returns a
  valid $a_{0}$ (respectively, $a_{1}$) \emph{conditioned} on $z$ and
  challenge $b=0$ (resp., $b=1$) are \emph{independent}, so both occur
  with probability at least $\alpha^{2}$.  All together then, the
  probability that $K$ outputs $\log_{g}(x)$ is at least
  $\alpha^{3}/2$, as claimed.

\end{proof}

\end{document}

%%% Local Variables:
%%% mode: latex
%%% TeX-master: t
%%% End:
