\documentclass[11pt]{article}

\usepackage{fullpage}
\usepackage{mathptmx}
\usepackage[T1]{fontenc}
\usepackage{hyperref,microtype,pdfsync}
\usepackage{amsmath,amsfonts,amssymb,amsthm}
\usepackage{fancyhdr}
\usepackage{mathtools}
\usepackage[noend]{algorithm,algorithmic}

%%% BLACKBOARD SYMBOLS

% hyperref package defines C and G, undefined to avoid conflicts
\let\C\relax
\let\G\relax
\newcommand{\C}{\ensuremath{\mathbb{C}}}
\newcommand{\D}{\ensuremath{\mathbb{D}}}
\newcommand{\F}{\ensuremath{\mathbb{F}}}
\newcommand{\G}{\ensuremath{\mathbb{G}}}
\newcommand{\J}{\ensuremath{\mathbb{J}}}
\newcommand{\N}{\ensuremath{\mathbb{N}}}
\newcommand{\Q}{\ensuremath{\mathbb{Q}}}
\newcommand{\R}{\ensuremath{\mathbb{R}}}
\newcommand{\T}{\ensuremath{\mathbb{T}}}
\newcommand{\Z}{\ensuremath{\mathbb{Z}}}
\newcommand{\QR}{\ensuremath{\mathbb{QR}}}

\newcommand{\Zt}{\ensuremath{\Z_t}}
\newcommand{\Zp}{\ensuremath{\Z_p}}
\newcommand{\Zq}{\ensuremath{\Z_q}}
\newcommand{\ZN}{\ensuremath{\Z_N}}
\newcommand{\Zps}{\ensuremath{\Z_p^*}}
\newcommand{\ZNs}{\ensuremath{\Z_N^*}}
\newcommand{\JN}{\ensuremath{\J_N}}
\newcommand{\QRN}{\ensuremath{\QR_{N}}}
\newcommand{\QRp}{\ensuremath{\QR_{p}}}

%%% THEOREM COMMANDS

\theoremstyle{plain}            % following are "theorem" style
\newtheorem{theorem}{Theorem}[section]
\newtheorem{lemma}[theorem]{Lemma}
\newtheorem{corollary}[theorem]{Corollary}
\newtheorem{proposition}[theorem]{Proposition}
\newtheorem{claim}[theorem]{Claim}
\newtheorem{fact}[theorem]{Fact}

\theoremstyle{definition}       % following are def style
\newtheorem{definition}[theorem]{Definition}
\newtheorem{conjecture}[theorem]{Conjecture}
\newtheorem{example}[theorem]{Example}
\newtheorem{protocol}[theorem]{Protocol}

\theoremstyle{remark}           % following are remark style
\newtheorem{remark}[theorem]{Remark}
\newtheorem{note}[theorem]{Note}
\newtheorem{exercise}[theorem]{Exercise}

% equation numbering style
\numberwithin{equation}{section}

%%% GENERAL COMPUTING

\newcommand{\bit}{\ensuremath{\set{0,1}}}
\newcommand{\pmone}{\ensuremath{\set{-1,1}}}

% asymptotics
\DeclareMathOperator{\poly}{poly}
\DeclareMathOperator{\polylog}{polylog}
\DeclareMathOperator{\negl}{negl}
\newcommand{\Otil}{\ensuremath{\tilde{O}}}

% probability/distribution stuff
\DeclareMathOperator*{\E}{E}
\DeclareMathOperator*{\Var}{Var}

% sets in calligraphic type
\newcommand{\calD}{\ensuremath{\mathcal{D}}}
\newcommand{\calF}{\ensuremath{\mathcal{F}}}
\newcommand{\calG}{\ensuremath{\mathcal{G}}}
\newcommand{\calH}{\ensuremath{\mathcal{H}}}
\newcommand{\calX}{\ensuremath{\mathcal{X}}}
\newcommand{\calY}{\ensuremath{\mathcal{Y}}}

% types of indistinguishability
\newcommand{\compind}{\ensuremath{\stackrel{c}{\approx}}}
\newcommand{\statind}{\ensuremath{\stackrel{s}{\approx}}}
\newcommand{\perfind}{\ensuremath{\equiv}}

% font for general-purpose algorithms
\newcommand{\algo}[1]{\ensuremath{\mathsf{#1}}}
% font for general-purpose computational problems
\newcommand{\problem}[1]{\ensuremath{\mathsf{#1}}}
% font for complexity classes
\newcommand{\class}[1]{\ensuremath{\mathsf{#1}}}

% complexity classes and languages
\renewcommand{\P}{\class{P}}
\newcommand{\BPP}{\class{BPP}}
\newcommand{\NP}{\class{NP}}
\newcommand{\coNP}{\class{coNP}}
\newcommand{\AM}{\class{AM}}
\newcommand{\coAM}{\class{coAM}}
\newcommand{\IP}{\class{IP}}

%%% "LEFT-RIGHT" PAIRS OF SYMBOLS

%% NOTE: this requires \usepackage{mathtools} in the document preamble

% inner product
\DeclarePairedDelimiter\inner{\langle}{\rangle}
% absolute value
\DeclarePairedDelimiter\abs{\lvert}{\rvert}
% a set
\DeclarePairedDelimiter\set{\{}{\}}
% parens
\DeclarePairedDelimiter\parens{(}{)}
% tuple, alias for parens
\DeclarePairedDelimiter\tuple{(}{)}
% square brackets
\DeclarePairedDelimiter\bracks{[}{]}
% rounding off
\DeclarePairedDelimiter\round{\lfloor}{\rceil}
% floor function
\DeclarePairedDelimiter\floor{\lfloor}{\rfloor}
% ceiling function
\DeclarePairedDelimiter\ceil{\lceil}{\rceil}
% length of some vector, element
\DeclarePairedDelimiter\length{\lVert}{\rVert}
% "lifting" of a residue class
\DeclarePairedDelimiter\lift{\llbracket}{\rrbracket}
\DeclarePairedDelimiter\len{\lvert}{\rvert}

%%% CRYPTO-RELATED NOTATION

% KEYS AND RELATED

\newcommand{\key}[1]{\ensuremath{#1}}

\newcommand{\pk}{\key{pk}}
\newcommand{\vk}{\key{vk}}
\newcommand{\sk}{\key{sk}}
\newcommand{\mpk}{\key{mpk}}
\newcommand{\msk}{\key{msk}}
\newcommand{\fk}{\key{fk}}
\newcommand{\id}{id}
\newcommand{\keyspace}{\ensuremath{\mathcal{K}}}
\newcommand{\msgspace}{\ensuremath{\mathcal{M}}}
\newcommand{\ctspace}{\ensuremath{\mathcal{C}}}
\newcommand{\tagspace}{\ensuremath{\mathcal{T}}}
\newcommand{\idspace}{\ensuremath{\mathcal{ID}}}

\newcommand{\concat}{\ensuremath{\|}}

% GAMES

% advantage
\newcommand{\advan}{\ensuremath{\mathbf{Adv}}}

% different attack models
\newcommand{\attack}[1]{\ensuremath{\text{#1}}}

\newcommand{\atk}{\attack{atk}} % dummy attack
\newcommand{\indcpa}{\attack{ind-cpa}}
\newcommand{\indcca}{\attack{ind-cca}}
\newcommand{\anocpa}{\attack{ano-cpa}} % anonymous
\newcommand{\anocca}{\attack{ano-cca}}
\newcommand{\euacma}{\attack{eu-acma}} % forgery: adaptive chosen-message
\newcommand{\euscma}{\attack{eu-scma}} % forgery: static chosen-message
\newcommand{\suacma}{\attack{su-acma}} % strongly unforgeable

% ADVERSARIES
\newcommand{\attacker}[1]{\ensuremath{\mathcal{#1}}}

\newcommand{\Adv}{\attacker{A}}
\newcommand{\AdvA}{\attacker{A}}
\newcommand{\AdvB}{\attacker{B}}
\newcommand{\Dist}{\attacker{D}}
\newcommand{\Sim}{\attacker{S}}
\newcommand{\Ora}{\attacker{O}}
\newcommand{\Inv}{\attacker{I}}
\newcommand{\For}{\attacker{F}}

% CRYPTO SCHEMES

\newcommand{\scheme}[1]{\ensuremath{\text{#1}}}

% pseudorandom stuff
\newcommand{\prg}{\algo{PRG}}
\newcommand{\prf}{\algo{PRF}}
\newcommand{\prp}{\algo{PRP}}

% symmetric-key cryptosystem
\newcommand{\skc}{\scheme{SKC}}
\newcommand{\skcgen}{\algo{Gen}}
\newcommand{\skcenc}{\algo{Enc}}
\newcommand{\skcdec}{\algo{Dec}}

% public-key cryptosystem
\newcommand{\pkc}{\scheme{PKC}}
\newcommand{\pkcgen}{\algo{Gen}}
\newcommand{\pkcenc}{\algo{Enc}} % can also use \kemenc and \kemdec
\newcommand{\pkcdec}{\algo{Dec}}

% digital signatures
\newcommand{\sig}{\scheme{SIG}}
\newcommand{\siggen}{\algo{Gen}}
\newcommand{\sigsign}{\algo{Sign}}
\newcommand{\sigver}{\algo{Ver}}

% message authentication code
\newcommand{\mac}{\scheme{MAC}}
\newcommand{\macgen}{\algo{Gen}}
\newcommand{\mactag}{\algo{Tag}}
\newcommand{\macver}{\algo{Ver}}

% key-encapsulation mechanism
\newcommand{\kem}{\scheme{KEM}}
\newcommand{\kemgen}{\algo{Gen}}
\newcommand{\kemenc}{\algo{Encaps}}
\newcommand{\kemdec}{\algo{Decaps}}

% identity-based encryption
\newcommand{\ibe}{\scheme{IBE}}
\newcommand{\ibesetup}{\algo{Setup}}
\newcommand{\ibeext}{\algo{Ext}}
\newcommand{\ibeenc}{\algo{Enc}}
\newcommand{\ibedec}{\algo{Dec}}

% hierarchical IBE (as key encapsulation)
\newcommand{\hibe}{\scheme{HIBE}}
\newcommand{\hibesetup}{\algo{Setup}}
\newcommand{\hibeext}{\algo{Extract}}
\newcommand{\hibeenc}{\algo{Encaps}}
\newcommand{\hibedec}{\algo{Decaps}}

% binary tree encryption (as key encapsulation)
\newcommand{\bte}{\scheme{BTE}}
\newcommand{\btesetup}{\algo{Setup}}
\newcommand{\bteext}{\algo{Extract}}
\newcommand{\bteenc}{\algo{Encaps}}
\newcommand{\btedec}{\algo{Decaps}}

% trapdoor functions
\newcommand{\tdf}{\scheme{TDF}}
\newcommand{\tdfgen}{\algo{Gen}}
\newcommand{\tdfeval}{\algo{Eval}}
\newcommand{\tdfinv}{\algo{Invert}}
\newcommand{\tdfver}{\algo{Ver}}

%%% PROTOCOLS

\newcommand{\out}{\text{out}}
\newcommand{\view}{\text{view}}

%%% COMMANDS FOR LECTURES/HOMEWORKS

\newcommand{\lecheader}{%
  \chead{\large \textbf{Lecture \lecturenum\\\lecturetopic}}

  \lhead{\small \textbf{Theory of Cryptography}\\}

  \rhead{\small \textbf{Instructor:
      \href{http://www.eecs.umich.edu/~cpeikert/}{Chris Peikert}\\Scribe:
      \scribename}}

  \setlength{\headheight}{20pt}
  \setlength{\headsep}{16pt}
}


% VARIABLES

\newcommand{\lecturenum}{17}
\newcommand{\lecturetopic}{Zero-Knowledge for NP}
\newcommand{\scribename}{Akash Kumar}

% END OF VARIABLES

\lecheader

\pagestyle{plain}               % default: no special header

\begin{document}

\thispagestyle{fancy}           % first page should have special header

% LECTURE MATERIAL STARTS HERE

\section{Recap: Zero-Knowledge Proofs}
\label{sec:recap:zkp}

\begin{definition}
  \label{def:ips}
  A \emph{zero-knowledge interactive proof system} with
  \emph{soundness error} $s \in [0,1]$ for a language $L \subseteq
  \bit^{*}$ is a pair of algorithms: a (possibly computationally
  unbounded) prover $P$, and a ppt verifier $V$, having the following
  properties:
  \begin{enumerate}
  \item \emph{Completeness}: for all $x \in L$, $\out_{V}[ P(x)
    \leftrightarrow V(x) ] = 1$, with probability $1$.
  \item \emph{Soundness}: for any $x \not\in L$ and any (possibly
    unbounded) $P^{*}$,
    \[ \Pr\left[ \out_{V}[ P^{*}(x) \leftrightarrow V(x) ] = 1 \right]
    \leq s. \]
  \item \emph{Zero-knowledgeness}: for every nuppt (possibly cheating)
    verifier $V^{*}$, there exists an nuppt simulator $\Sim$ such that
    for all $x \in L$, \[ \view_{V^{*}}[P(x) \leftrightarrow V^{*}(x)]
    \approx \Sim(x). \]
  \end{enumerate}

  The type of indistinguishability in the zero-knowledge condition can
  either be \emph{statistical} (i.e., $\statind$) or
  \emph{computational} (i.e., $\compind$).  We also say that the proof
  system is \emph{efficient} if the prover's strategy can be
  implemented by a randomized algorithm $P(x,w)$ that runs in time
  polynomial in the length of its first argument, and any $x \in L$
  has a ``hint'' (or witness) $w$ for which $P$ functions as it
  should.
\end{definition}

\section{Which Languages Have ZKPs?}
\label{sec:which-languages-zkp}

We've seen zero-knowledge (sometime just honest-verifier ZK) proofs
for graph non-isomorphism and graph isomorphism.  There are similar
proofs for quadratic residuosity and non-residuosity modulo $N=pq$.
(As an exercise, try to design them.)  What other languages have
zero-knowledge proofs?  If we had to start from scratch every time we
wanted a ZKP for a new language, it would be very cumbersome and
time-consuming.  Here, we will see a very powerful and general result
of Goldreich, Micali and Wigderson: informally, it says that ``any
language for which membership can be efficiently verified can be
proved in zero-knowledge.''

\begin{theorem}
  \label{thm:zkp-np}
  Assume that there exists a statistically binding / computationally
  hiding commitment scheme, defined below.  In particular, such a
  commitment scheme can be constructed from any one-way permutation,
  or even any one-way function.  Then for any $\NP$-language $L$ (also
  defined below), there is a zero-knowledge proof system (with
  efficient prover) for $L$.  
\end{theorem}

\subsection{Preliminaries}
\label{sec:preliminaries}

\subsubsection{$\NP$ and $\NP$-hardness}
\label{sec:np}

For completeness, we recall the definition of the class $\NP$ of
languages that are decidable in nondeterministic polynomial time.

\begin{definition}
  \label{def:np}
  A language $L \subseteq \bit^{*}$ is said to be in $\NP$ if there
  exists a deterministic algorithm $W(x,w)$, running in time
  polynomial in the length of its first argument $x$, such that $x \in
  L$ if and only if there exists some $w \in \bit^{*}$ for which
  $W(x,w)$ accepts.
\end{definition}

(Note that even though $W$ is a deterministic machine, its second
argument $w$ captures the nondeterminism in the definition.)  If
$W(x,w)$ accepts, we say that $w$ is a ``witness'' (or a
``certificate,'' or a ``proof'') that $x \in L$.  Recall that any
language $L \in \P$ (i.e., one for which membership can be decided in
\emph{deterministic} polynomial time) is also in $\NP$: the machine
$W(x,w)$ can just ignore its second argument and decide whether $x \in
L$ on its own.  But $\NP$ also contains many languages that are not
believed to be in $\P$, such as $\problem{3SAT}$, the language of all
satisfiable 3CNF formulas; a witness $w$ is a satisfying assignment
for the given formula.

Also recall that $L'$ is said to be an \emph{$\NP$-hard} language if
for \emph{every} $\NP$-language $L$, there is a deterministic
polynomial-time algorithm (a reduction) $R_{L}$ such that $x \in L$ if
and only if $R_{L}(x) \in L'$.  A language $L'$ is said to be
\emph{$\NP$-complete} if it is $\NP$-hard, and is itself in $\NP$.
Many important problems are $\NP$-complete, including
$\problem{3SAT}$, graph $3$-colorability (defined below), graph
Hamiltonicity, and so on.

\subsubsection{Commitment Scheme}
\label{sec:commitment-scheme}

\newcommand{\Com}{\algo{Com}}

Consider the following scenario.  Alice and Bob are betting on the
result of a coin toss: Bob tosses the coin and Alice calls it while
it's still in the air.  This game is trivial to analyze.  But what if
Alice and Bob are not at the same location --- what if they are
playing this game over the phone?  If Alice announces her call first,
then Bob can lie about the outcome of his coin toss and always win.
If Bob announces the outcome first, then Alice can just call the same
thing and always win.  In order to ensure fair play, we want Alice to
\emph{commit} to her call before the coin flip, in a way that conceals
her choice from Bob but does not allow her to change her call after
Bob announces the outcome of the coin flip.  The idea is that Alice
can run a randomized algorithm $\Com$ on her guess, then later reveal
it and the random coins she used when running $\Com$.  This should
allow Bob to check that she revealed her commitment correctly, while
hiding the value of her guess until it is revealed.  A formal
definition follows.

\begin{definition}
  A statistically binding, computationally hiding \emph{commitment
    scheme} for message space $\bit^{\ell}$ is a ppt algorithm $\Com$
  having the following properties:
  \begin{itemize}
  \item \emph{Statistical (perfect) binding}: for all \emph{distinct}
    $m_0, m_1 \in \bit^{\ell}$ and all random coins $r_0, r_1 \in
    \bit^{*}$, \[ \Com(m_0; r_0) \neq \Com(m_1; r_1). \]
   
  \item \emph{Computational hiding}: For all messages $m_0, m_1 \in
    \bit^{\ell}$, over the random coins of $\Com$, \[ \Com(m_{0})
    \compind \Com(m_{1}). \]
  \end{itemize}
\end{definition}

\begin{remark}
  \label{rem:com-enc}
  A commitment scheme is much like a public-key encryption scheme, in
  that it computationally hides the committed message, but commitment
  is potentially weaker.  The reason is that there need not be any
  \emph{decryption} algorithm; instead, the message and randomness are
  revealed by Alice at a later point.
\end{remark}

It is simple to construct a commitment scheme for a \emph{single bit}
using a one-way \emph{permutation} $f$ with a hard-core predicate $h$.
Define $\Com(b \in \bit, r \in \bit^n) = (f(r), b \oplus h(r))$.  The
scheme is clearly \textit{binding}: given any commitment $(y \in
\bit^{n}, c \in \bit)$, there is a unique $r = f^{-1}(y)$ and hence
the committed bit is $b = c \oplus h(r)$.  That the above scheme is
also \emph{hiding} follows by noticing that $(f(r), h(r)) \compind
(f(r), U_1) \compind (f(r), 1 \oplus h(r))$.  To extend the scheme to
multiple-bit messages, just execute $\Com$ in parallel with
independent randomness on each bit; hiding follows by a simple hybrid
argument.

In fact, as we have seen in the homework, it is possible to construct
an \emph{interactive} (two-message) commitment scheme using a
pseudorandom generator, which can be constructed (with a lot of work)
from any one-way \emph{function}.

\section{Zero-Knowledge for $\NP$}
\label{sec:zero-knowledge-np}

\newcommand{\tcol}{\problem{3COL}}

We can now prove Theorem~\ref{thm:zkp-np}.  For the reasons given
below, it will suffice to give a zero-knowledge proof (with efficient
prover) for the \emph{graph $3$-colorability} problem $\tcol$, defined
as follows.

\begin{definition}
  \label{def:3col}
  We say that a graph $G = (V,E)$ is \emph{$3$-colorable} if there
  exists a function $\pi \colon V \to \set{1,2,3}$ (a ``coloring'' of
  the vertices) such that $\pi(i) \neq \pi(j)$ for every edge $(i,j)
  \in E$; that is, the endpoints of each edge have distinct colors.
  The language $\tcol$ is the set of graphs $G$ that are
  $3$-colorable.
\end{definition}

It suffices to give a zero-knowledge proof for $\tcol$ because $\tcol$
is $\NP$-complete.  In more detail, to get a ZKP for any language $L
\in \NP$, the prover and verifier (and also the simulator) on input
$x$ first compute $x' = R_{L}(x)$, where $R_{L}$ is a deterministic
poly-time reduction from $L$ to $\tcol$; such a reduction is
guaranteed to exist because $\tcol$ is $\NP$-complete.  (In fact, the
Cook-Levin reduction, which works for any $L \in \NP$, is a
``canonical'' reduction that can be used.)  Then the prover and
verifier run the ZKP for $\tcol$ on $x' = R_{L}(x)$; because $R_{L}(x)
\in \tcol$ if and only if $x \in L$, this yields a ZKP for $L$.  Note
that to get an \emph{efficient} prover for $L$, we also need an
efficient reduction that maps any \emph{witness} for $x \in L$ to a
witness for $R_{L}(x) \in \tcol$; fortunately, the Cook-Levin
reduction provides this as well.

\subsection{Interactive Proof for $\tcol$}
\label{sec:ip-3col}

Let $G = ([n],E)$ be a graph on $n$ vertices, which are numbered $1,
\ldots, n$ without loss of generality.

Consider the setup again. $P$ wants to convince $V$ that there is a
valid $3$-coloring $\pi \colon [n] \to \set{1,2,3}$ of the graph $G$.
The prover and verifier are both given the graph $G$, and the prover
is additionally given $\pi$.  The basic idea is that the prover
commits to a \emph{random permutation} of the coloring $\pi$, then the
verifier challenges the prover to reveal the colors of the two
vertices on a single edge.  If the prover correctly reveals two
different colors, then the verifier accepts.  Intuitively, the
protocol is sound because if the graph is not $3$-colorable, the
verifier has at least a $1/\abs{E}$ chance of choosing an improperly
colored edge.  Again intuitively, the protocol is zero-knowledge
because the verifier only sees the commitments to each vertex (which
hide the colors), plus openings to two distinct random colors.  The
formal protocol follows.

\begin{center}
 \begin{tabular}{ccc}
   $P(G, \pi)$ & & $V(G)$ \\ \\
   Choose random perm $\rho$ of $\set{1,2,3}$. & & \\
   For all $i \in [n]$, let $k_{i} = \rho(\pi(i))$ & & \\
   and let $c_{i} = \Com(k_{i}; r_{i})$. & & \\
   & $\underrightarrow{\quad c_{1}, \ldots, c_{n} \quad }$ & \\
   && Choose $(u,v) \gets E$ \\
   & $\underleftarrow{\quad (u,v) \quad}$ & \\
   If $(u,v) \in E$, && \\
   & $\underrightarrow{\quad (k_{u}, r_{u}), (k_{v}, r_{v})
     \quad}$ & \\
   && Accept iff $k_{u}, k_{v} \in \set{1,2,3}$ are distinct \\
   && and $c_{i} = \Com(k_{i}, r_{i})$ for $i=u,v$.
 \end{tabular}
\end{center}

The following claim is immediate by inspection:

\begin{claim}
  \label{claim:complete}
  The interactive proof system $(P,V)$ described above is complete.
\end{claim}

\begin{claim}
  \label{claim:sound}
  The proof system $(P,V)$ described above has soundness error at most
  $1-1/\abs{E} \leq 1-1/n^{2}$.
\end{claim}

\begin{proof}
  Suppose that $G$ is not $3$-colorable, and let $P^{*}$ be any
  possibly unbounded prover.  By statistical binding of $\Com$, for
  any value of a commitment $c_{i}$ sent by $P^{*}$, there is at most
  one color $k_{i} \in \set{1,2,3}$ for which $P^{*}$ could acceptably
  open the commitment $c_{i}$.  Because $G$ is not $3$-colorable, for
  any values of the colors $k_{i} \in \set{1,2,3}$, there must be at
  least one edge $(u,v) \in E$ for which $k_{u} = k_{v}$.  The
  verifier chooses this edge with probability $1/\abs{E}$, and no
  message from $P^{*}$ could cause it to accept in this case.
\end{proof}

\begin{claim}
  \label{claim:zk}
  The proof system $(P,V)$ described above is computational
  zero-knowledge.
\end{claim}

\begin{proof}
  Let $G \in \tcol$, and let $\pi$ be an arbitrary valid $3$-coloring
  of $G$.  Let $V^{*}$ be an arbitrary nuppt cheating verifier.  We
  need to show that there exists an nuppt simulator $\Sim$ such
  that \[\view_{V^{*}}[P(G,\pi) \leftrightarrow V^{*}(G)] \compind
  \Sim(G). \] As usual, we will give a ``black-box'' simulator $\Sim$
  that uses $V^{*}$ only as an oracle.  Informally, the simulator
  prepares an initial message to $V^{*}$ (i.e., the commitments
  $c_{i}$) so that it knows how to correctly answer \emph{one} of the
  possible challenge edges.  If $V^{*}$ happens to ask for that edge,
  the simulator completes the transcript with the correct reply,
  otherwise it rewinds and tries again until it succeeds (up to a
  certain number of attempts).  More formally, the simulator
  $\Sim^{V^{*}}(G)$ is defined as follows:
  \begin{algorithmic}[1]
    \STATE Choose $(u, v) \gets E$ uniformly at random.
    
    \STATE Choose \emph{distinct} uniformly random $k_{u}, k_{v} \in
    \set{1,2,3}$.  For all other $i \in [n]$, set $k_{i} = 1$.
    
    \STATE For each $i \in [n]$, compute commitments $c_{i} =
    \Com(k_{i}; r_{i})$.

    \STATE Run $V^{*}$ with fresh random coins $r_{V^{*}}$, send the
    $c_{i}$s to $V^{*}$, and receive a challenge edge $(u^{*},v^{*})
    \in E$.
    
    (If $V^{*}$ does not reply with a valid edge, then just
    output the view so far.)

    \IF{ $(u^{*}, v^{*}) = (u,v)$ }
    
    \STATE Output the view $(G, r_{V^{*}}, \set{c_{i}}, (k_{u},
    r_{u}), (k_{v}, r_{v}))$.

    \ELSE

    \STATE Restart from the beginning (up to $n^{3}$ total
    iterations).

    \ENDIF
  \end{algorithmic}
  
  To prove that $\Sim$ is a valid simulator, i.e., that
  $\view_{V^{*}}[P(G,\pi) \leftrightarrow V^{*}(G)] \compind \Sim(G)$,
  we will consider a sequence of hybrid experiments that successively
  change from the view in a true interaction with $P$ to the view
  output by $\Sim$.  The hybrid experiments are defined as follows,
  along with the reasons why successive hybrids are indistinguishable,
  which will complete the proof.
  \begin{itemize}
  \item $H_{0}$ is the view of $V^{*}$ in an interaction with the real
    prover $P(G,\pi)$.
  \item $H_{1}$ is the view of $V^{*}$ in an interaction with an
    algorithm $P'(G,\pi)$.  The algorithm $P'$ works almost
    identically to $P$, but it first chooses an edge $(u,v) \gets E$
    before preparing the commitments $c_{i}$ as usual.  It then
    replies to $V^{*}$'s challenge $(u^{*},v^{*})$ \emph{only if}
    $(u^{*}, v^{*}) = (u,v)$; otherwise, the experiment is re-run from
    the beginning (with fresh random coins), up to $n^{3}$ iterations.
    In other words, $P'(G,\pi)$ generates messages exactly as
    $P(G,\pi)$ does, but has the rewinding strategy of $\Sim$.

    Observe that conditioned on the event $(u^{*}, v^{*}) = (u,v)$,
    the view of $V^{*}$ when interacting with $P'$ is identical to its
    view when interacting with $P$.  And because the choice of $(u,v)$
    by $P'$ is \emph{statistically independent} of its initial message
    to $V^{*}$, the probability that $(u^{*},v^{*}) = (u,v)$ is
    exactly $1/\abs{E} \geq 1/n^{2}$, and some iteration succeeds,
    except with negligible probability.  We conclude that $H_{0}
    \statind H_{1}$.

  \item $H_{2}$ is the view of $V^{*}$ in an interaction with an
    algorithm $P''(G,\pi)$.  The algorithm $P''$ works almost
    identically to $P'$: it chooses $(u,v) \gets E$ and sets the
    colors $k_{u} = \rho(\pi(u))$, $k_{v} = \rho(\pi(v))$ as before,
    but sets all other colors $k_{i} = 1$ (ignoring the valid coloring
    $\pi$).  It then constructs the commitments $c_{i}$ using the
    colors $k_{i}$, and either replies to $V^{*}$ or rewinds in
    exactly the same way as before.  In other words, $P''(G,\pi)$
    still uses the valid coloring $\pi$ for vertices $u$ and $v$, but
    uses $\Sim$'s coloring strategy for the others.

    Using the computational hiding property of the commitment scheme,
    one can show that $H_{1} \compind H_{2}$.  To do this formally, we
    would actually consider a sequence of ``sub-hybrids'' in which one
    color $k_{i}$ at a time is changed from its ``true'' value
    $\rho(\pi(i))$ (as in $H_{1}$) to its ``simulated'' value $1$ (as
    in $H_{2}$).  We remark that the reduction from breaking the
    commitment scheme to distinguishing adjacent sub-hybrids uses
    \emph{non-uniformity} in an essential way, because to simulate the
    entire view of $V^{*}$ (given a commitment to either
    $\rho(\pi(i))$ or $1$), the simulator needs to have $G$ and $\pi$
    ``hard-coded'' into it.  (This is a rather technical point, but is
    an important one for a rigorous proof.)

  \item $H_{3}$ is the view of $V^{*}$ in an interaction with
    $\Sim(G)$.  It is easy to see that $H_{2}$ and $H_{3}$ are
    \emph{identical}: for any valid $3$-coloring $\pi$, over the
    random choice of $\rho$ by $P''$, the values $k_{u} =
    \rho(\pi(u))$ and $k_{v} = \rho(\pi(v))$ are \emph{distinct
      random} colors, independent of everything else (just as in the
    definition of $\Sim$). \qedhere
  \end{itemize}
\end{proof}

\section{Applications of ZK}
\label{sec:applications-zk}

We conclude by mentioning a few fascinating applications of zero
knowledge proofs for $\NP$.

\paragraph{Deniability / Non-transferability.}

Suppose that a CIA agent Alice wants to prove her credentials to an
MI5 agent Bob.  One way to do this is for the CIA to give her a
signature, signed under the CIA's public key, for the message ``Alice
is a CIA agent.''  Alice can then give this signature to Bob.
However, what if Bob goes rogue and wants to sell his list of known
CIA agents (with proof, in the form of valid signatures!) to the
highest bidder?  The problem is that signature identifying Alice as a
CIA agent is \emph{transferable}.  Instead, Alice should prove to Bob
using zero-knowledge that she has a signature for the message ``Alice
is a CIA agent.''  This is an $\NP$-statement, since the signature is
a valid witness (which Alice has in her possession).  Bob will believe
the proof, but he will not be able to convincingly transfer the
transcript of that proof to anybody else, because for all they know,
Bob could have created the transcript on his own by running the
simulator!  In other words, Alice's proof is \emph{deniable}, in that
she can plausibly claim that she was not responsible for producing it.

\paragraph{Enforcing honest behavior.}

Here the idea is that players in an interactive protocol can prove
that they are not ``cheating,'' i.e., that they are not deviating from
prescribed (and known) programs that they are supposed to be running
(on their own secret inputs).

Roughly, the idea is that at the outset of the protocol, the players
are required to commit to the secret inputs and random coins of the
``honest'' (prescribed) algorithms that they are supposed to run.
They then carry out their computations, and with each output message
they prove to each other in zero-knowledge that the message was
honestly computed under the committed inputs and coins.  By soundness,
we know that the players must really act honestly in order to be able
to provide a valid proof.  By zero-knowledge, the proofs do not
compromise the privacy of their secret inputs.

\end{document}

%%% Local Variables: 
%%% mode: latex
%%% TeX-master: t
%%% End: 
